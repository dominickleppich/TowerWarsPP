\section*{Quellcode}
\subsection*{Preset}
Die Klasse \code{Position}, die Positionen auf dem Spielbrett beschreibt.
\lstinputlisting{files/preset/Position.java}

Die Klasse \code{Move}, die Spielzüge beschreibt.
\lstinputlisting{files/preset/Move.java}

Exceptions die beim Erzeugen von Positionen und Zügen aus Strings geworfen werden können.
\lstinputlisting{files/preset/PositionFormatException.java}
\lstinputlisting{files/preset/MoveFormatException.java} \newpage

Interface für den Spieler und Enumerations für Spielerfarbe, Spielertyp und Status.
\lstinputlisting{files/preset/Player.java}
\lstinputlisting{files/preset/PlayerColor.java}
\lstinputlisting{files/preset/PlayerType.java}
\lstinputlisting{files/preset/Status.java}

Die Interfaces zum Kapseln der Ein- und Ausgabe.
\lstinputlisting{files/preset/Requestable.java}
\lstinputlisting{files/preset/Viewable.java}
\lstinputlisting{files/preset/Viewer.java}

\newpage
\subsection*{Argumentparser}
Der ArgumentParser mit seiner Exception.
\lstinputlisting{files/preset/ArgumentParser.java}
\lstinputlisting{files/preset/ArgumentParserException.java}

\newpage
Ein Beispiel zur Verwendung des Parsers.
\begin{lstlisting}
package towerwarspp.demo;

import towerwarspp.preset.ArgumentParser;
import towerwarspp.preset.ArgumentParserException;

public class ArgumentParserTest {
    public static void main(String[] args) {
        try {
            ArgumentParser ap = new ArgumentParser(args);

            System.out.println("size: " + ap.getSize());

            switch (ap.getRed()) {
                case HUMAN:
                    System.out.println("Red chose human player");
                    break;
                case RANDOM_AI:
                    System.out.println("Red chose random player");
                    break;

                // and so on..
            }
        } catch (ArgumentParserException e) {
            // Something went wrong...
            e.printStackTrace();
        }
    }
}

\end{lstlisting}

\newpage
\subsection*{Netzwerk (RMI)}
Ein Gerüst für einen Netzwerkspieler finden Sie nachfolgend.
\begin{lstlisting}
package towerwarspp.demo;

import java.rmi.*;
import java.rmi.server.UnicastRemoteObject;
import java.net.*;

import towerwarspp.preset.*;

public class NetPlayer extends UnicastRemoteObject implements Player {
	private Player player;
	
	// ---------------------------------------------
	
	public NetPlayer(Player player) throws RemoteException {
		this.player = player;
	}
	
	public Move request() throws Exception, RemoteException {
		return player.request();
	}
	
	public void update(Move opponentMove, Status boardStatus)
		throws Exception, RemoteException {
		player.update(opponentMove, boardStatus);
	}
	
	public void confirm(Status boardStatus)
		throws Exception, RemoteException {
		player.confirm(boardStatus);
	}		
	
	public void init(int size, PlayerColor color)
		throws Exception, RemoteException {
		player.init(size, color);	
	} 
}
\end{lstlisting}

Die folgende Methode \code{offer} ist ein Beispiel wie ein Netzwerkspieler an einen Rechner \code{host}, auf dem eine RMI-Registry läuft, unter einem beliebigen Namen \code{name} angeboten werden kann.
\begin{lstlisting}
	public void offer(Player p, String host, String name) {
		try {
			Naming.rebind("rmi://" + host + "/" + name, p);
			System.out.println("Player (" + name + ") ready");
		} catch (MalformedURLException ex) {
			ex.printStackTrace();
		} catch (RemoteException ex) {
			ex.printStackTrace();
		}
	}
\end{lstlisting}

Die folgende Methode \code{find} ist ein Beispiel wie ein Spieler auf einen Rechner \code{host}, auf dem eine RMI-Registry läuft, unter einem bekannten Namen \code{name} gefunden werden kann.
\begin{lstlisting}
	public Player find(String host, String name) {
		Player p = null;
		try {
			p = (Player) Naming.lookup("rmi://" + host + "/" + name);
			System.out.println("Player (" + name + ") found");
		} catch (Exception ex) {
			ex.printStackTrace();
		}
		return p;
	}
\end{lstlisting}
