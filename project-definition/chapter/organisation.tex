\section*{Organisation}
\begin{enumerate}
\item Wenn Sie an der Prüfung zum Modul \emph{B.Inf.1802: Programmierpraktikum} teilnehmen möchten, müssen Sie sich bis zum \textbf{xx.xx.xxxx} in \textbf{FlexNow} anmelden.
\item Bilden Sie Projektgruppen mit vier Teilnehmern, größere Gruppen müssen ausdrücklich genehmigt werden und bekommen zusätzliche Aufgaben.

%Sie bilden \textbf{nicht} automatisch mit den Teilnehmern Ihrer StudIP-Veranstaltung eine Projektgruppe.
\item Vereinbaren Sie mit einem Tutor, der Ihr Projekt betreuen soll, einen Termin für ein regelmäßiges Treffen.
\item Bestimmen Sie einen Projektleiter. Zu den Aufgaben des Projektleiters gehört es die Entwicklung des Projekts als Ganzes zu steuern.
\begin{itemize}
\item Werden alle Anforderungen erfüllt?
\item Sind Sprache und Stil der Dokumentation einheitlich?
\item Ist eine überarbeitete Klasse von einem weiteren Projektmitglied kontrolliert worden?
\item \dots
\end{itemize}
\item Geben Sie der Projektgruppe einen aussagefähigen Namen ungleich \code{breaktroughPP}.
\item Melden Sie Ihre Projektgruppe an, unter Angabe des Namens und der Teilnehmer 
(E-Mail an \code{brosenne@informatik.uni-goettingen.de}).
\item Wählen Sie für die Prüfung einen Termin im Testzeitraum und lassen Sie sich diesen Termin bestätigen (E-Mail an \code{brosenne@informatik.uni-goettingen.de}).
\end{enumerate}