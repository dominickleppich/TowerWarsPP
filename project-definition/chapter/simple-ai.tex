\hypertarget{strategy}{\section*{Einfache Strategie}}
\underline{In Richtung gegnerischer Basis ziehen}

Es ist eine gute Idee in Richtung der gegnerischen Basis zu ziehen und weniger gut sich von ihr wegzubewegen. Für jeden Zug sei $d_1$ der Abstand zur gegnerischen Basis vor dem Zug und $d_2$ der Abstand nach dem Zug. Daraus berechnet sich ein Basis-Abstandsbonus \[r_\text{Basis} = d_1 - d_2\]

\underline{Steine und Türme schlagen}

Weiterhin ist es klug gegnerische Einheiten zu schlagen. Berechne hierfür einen Bonus für das Schlagen von gegnerischen Einheiten \[b_\text{kick} = \begin{cases}
0, & \text{wenn keine Einheit geschlagen wurde} \\
1, & \text{wenn ein Stein geschlagen wurde} \\
2, & \text{wenn ein Turm geschlagen wurde} \\
1337, & \text{wenn die gegnerische Basis geschlagen wurde}
\end{cases}\]

\underline{Gesamtwertung}

Berechnen Sie daraus für jeden Zug eine Bewertung
\[score = r_\text{Basis} + b_\text{kick}\]

und wählen Sie unter den Zügen mit Maximalbewertung zufällig einen aus.

\bigskip

Zusätzlich gelten die beiden folgenden Regeln.
\begin{enumerate}[label=\alph*)]
\item Gibt es einen Zug mit dem das Spiel gewonnen werden kann, dann wird dieser ausgeführt.
\item Ein Zug, der zur Folge hätte, dass der Gegner im nächsten Zug gewinnen könnte, wird dieser nur ausgeführt, wenn es keine Alternative gibt.
\end{enumerate}