\section*{Prüfung}
Nach der Abgabe wird das Projekt als Ganzes bewertet.

Während der Prüfung stellt jeder Teilnehmer den Teil des Projektes vor, für dessen Implementation er verantwortlich ist. Besonders interessant sind die aufgetretenen Probleme und deren Lösungen.

Neben der korrekten Umsetzung der Projektanforderungen wird gut lesbar und strukturierter Quellcode erwartet. Es sollten die Grundlagen des objektorientierten Programmentwurfs (z.B. Kapselung, Vererbung, Polymorphismus) berücksichtigt und die Möglichkeiten von Java (z.B. \emph{Java Collections Framework} ausgenutzt werden.

Jedem Teilnehmer werden Fragen zum Projekt, sowie zu Java, JavaDoc, Subversion und Ant, im Rahmen von Vorlesung und Übung, gestellt. Daraus ergibt sich für jeden Teilnehmer eine Tendenz (z.B. etwas schwächer als das Projekt insgesamt) und letztlich eine individuelle Note.