\section*{Organisation}
\begin{enumerate}
\item Wenn Sie an der Prüfung zum Modul \emph{B.Inf.1802: Programmierpraktikum} teilnehmen möchten, müssen Sie sich in \textbf{FlexNow} anmelden.

\textbf{An- und Abmeldefrist} für die Prüfung ist der \textbf{16.07.2017}.
\item Bilden Sie Projektgruppen mit vier Teilnehmern, größere Gruppen müssen ausdrücklich genehmigt werden und bekommen zusätzliche Aufgaben.
\item Vereinbaren Sie mit einem Tutor, der Ihr Projekt betreuen soll, einen Termin für ein regelmäßiges Treffen.
\item Bestimmen Sie einen Projektleiter. Zu den Aufgaben des Projektleiters gehört es die Entwicklung des Projekts als Ganzes zu steuern.
\begin{itemize}
\item Werden alle Anforderungen erfüllt?
\item Sind Sprache und Stil der Dokumentation einheitlich?
\item Ist eine überarbeitete Klasse von einem weiteren Projektmitglied kontrolliert worden?
\item \dots
\end{itemize}
\item Geben Sie der Projektgruppe einen aussagefähigen Namen ungleich \code{TowerWarsPP}.
\item Der Projektleiter meldet die Projektgruppe an, per E-Mail an \\\href{mailto:brosenne@informatik.uni-goettingen.de}{\code{brosenne@informatik.uni-goettingen.de}} mit Betreff \code{TowerWarsPP}.

In der E-Mail müssen Projektname, sowie die Namen und Matrikelnummern der Teilnehmer übermittelt werden.
\item Der Projektleiter reserviert einen Prüfungstermin, in Absprache mit den Mitgliedern der Projektgruppe und dem \textbf{Tutor}, unter \textit{Terminvergabe\textrightarrow APP-Prüfung} in der Stud.IP-Veranstaltung \textit{Allgemeines Programmierpraktikum} (oder im Profil des Stud.IP-Benutzers \textit{Rechnerübung Informatik}).

Bei der Reservierung des Prüfungstermins muss der Projektname angegeben werden.
\end{enumerate}