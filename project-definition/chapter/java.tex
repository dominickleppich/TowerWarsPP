\section*{Spiel}
Realisieren Sie \textbf{TowerWarsPP} als Computerspiel in Java. Kommentieren Sie den Quellcode ausführlich. Verwenden Sie JavaDoc für das \emph{Application Programming Interface (API)} und kommentieren Sie sonst wie üblich.

Die vorgegebenen Quelltexte finden Sie in der Stud.IP-Veranstaltung unter \\
\textit{Dateien\textrightarrow Projekt\textrightarrow TowerWarsPP.tgz}

Alle vorgegebenen Klassen sind im Anhang zu finden.

\begin{enumerate}
% Setup
\item Alle Klassen und Schnittstellen gehören zu einem Package, das mit \code{towerwarspp} beginnt.

Die vorgegebenen Klassen und Schnittstellen des Package \code{towerwarspp.preset} dürfen, mit Ausnahme von Kommentaren, nicht verändert werden. 

Es ist allerdings erlaubt folgende Klassen zu erweitern (nähere Informationen sind dem entsprechenden Abschnitt der Projektbeschreibung zu entnehmen):
\begin{description}
\item[\code{Viewer}] Darf komplett selbst geschrieben werden. Im Anhang befindet sich ein Viewer, der lediglich als Anregung dienen soll.
\item[\code{ArgumentParser}] Darf um weitere Parameter ergänzt werden.
\end{description}

\newpage
\subsection*{Spielbrett}

% Board
\item Erstellen Sie eine Spielbrett-Klasse mit folgenden Merkmalen:
\begin{itemize}
\item Ein Spielfeld mit $n \times n$ Feldern wird verwaltet. Dabei gilt $4 \le n \le 26$.
\item Implementieren Sie eine Methode, mit der Spielzüge entgegengenommen werden können.
\item Gültige Spielzüge und Spielzüge die zum Ende des Spiels führen werden erkannt.

Ein leerer Zug (\code{null}) ist gültig und wird als Aufgabe des Spielers gewertet.
\item Der erste entgegengenommene Spielzug gehört immer zum roten Spieler.
\item Ein Spielzug ist ein Objekt der Klasse \code{towerwarspp.preset.Move}, das mit Referenzen auf Objekte der Klasse \code{towerwarspp.preset.Position} arbeitet.

\item Implementieren Sie eine Methode, die einen Wert aus der Enumeration\\ \code{towerwarspp.preset.Status} zurückliefert, über die der Spielstand erfragt werden kann.

\item Die Schnittstelle \code{towerwarspp.preset.Viewable} wird implementiert.
\end{itemize}

% AI Board
\item Erstellen Sie eine erweiterte Spielbrett-Klasse, die von der vorherigen erbt und folgende Zusatzanforderungen erfüllt:

\begin{itemize}
\item Implementieren Sie eine Methode, die -- ausgehend von der aktuellen Spielsituation -- alle möglichen gültigen Spielzüge eines Spielers zurückliefert.

\item Implementieren Sie eine Methode, die Spielzüge anhand der unten beschriebenenen \hyperlink{strategy}{einfachen Strategie} bewertet.
\end{itemize}

\subsection*{Ein- und Ausgabe}

% Viewer
\item Erstellen Sie eine Klasse, die die Schnittstelle \code{towerwarspp.preset.Viewer} implementiert.

Diese Klasse soll es ermöglichen alle für das Anzeigen eines Spielbrett-Objekts nötigen Informationen zu erfragen, ohne Zugriff auf die Attribute des Spielbrett-Objekts zuzulassen.

Die Methode \code{viewer} des Spielbretts liefert ein passendes Objekt dieser Klasse.

% Text Input
\item Erstellen Sie eine Text-Eingabe-Klasse, die die Schnittstelle \\ \code{towerwarspp.preset.Requestable} implementiert.

Die Methode \code{deliver} fordert einen Zug, in einer Zeile, von der Standardeingabe an und liefert ein dazu passendes \code{towerwarspp.preset.Move}-Objekt zurück.

Verwenden Sie die statische Methode \code{parseMove} der Klasse \code{Move} um den von der Standardeingabe eingelesenen String in ein Move-Objekt umzuwandeln.

Die Methode \code{parseMove} wirft eine \code{towerwarspp.preset.MoveFormatException}, falls das Einlesen missglücken sollte. Auf diese Exception muss sinnvoll reagiert werden.

% Text Output
\item Erstellen Sie eine Text-Ausgabe-Klasse mit der der aktuelle Spielstand eines Spielbretts auf die Standardausgabe geschrieben werden kann. Dazu wird ein Objekt einer Klasse, die die Schnittstelle \code{towerwarspp.preset.Viewer} implementiert, benutzt.

Sehen Sie für die Ausgabe eine Schnittstelle vor die sich an \code{Requestable} orientiert und implementieren Sie dieses in der Text-Ausgabe-Klasse.

\underline{Hinweis}

Text-Eingabe- und Ausgabe-Klasse können in einer Klasse vereint werden.

% Graphic
\item Erstellen Sie eine grafische Ein-Ausgabe-Klasse. Diese Klasse implementiert die Schnittstellen \code{towerwarspp.preset.Requestable} und die von Ihnen geschriebene Ausgab-eSchnittstelle und benutzt ein Objekt einer Klasse, die die Schnittstelle \code{towerwarspp.preset.Viewer} implementiert, für eine einfache grafische Ausgabe.

\textbf{Sorgen Sie dafür, dass die Darstellung des Spielbretts der Größe des Fensters angepasst ist und beim Verändern der Fenstergröße mitskaliert.} \marginpar{\textbf{|}}

% Gültige Züge sind nur im erweiterten Board enthalten. Hier könnte es ein Problem geben. Vielleicht müssen wir es dieses mal leider doch weglassen oder die gültigen Züge ins normale Board verschieben...
\textbf{Sehen Sie eine Möglichkeit vor die gültigen Züge grafisch darstellen zu können.}\marginpar{\textbf{|}}

\underline{Hinweis}

Investieren Sie nicht zu viel Zeit in das Design, denn es wird nicht bewertet.

Sie können nützliche Informationen zum Zeichnen und Arbeiten mit hexagonalen Spielfeldern auf dieser Seite finden: \href{http://www.redblobgames.com/grids/hexagons}{http://www.redblobgames.com/grids/hexagons}

\subsection*{Spieler}

% Player
\item Alle Spieler implementieren die Schnittstelle \code{towerwarspp.preset.Player}.

\begin{itemize}
\item Ein Spieler hat keine Referenz auf das Spielbrett-Objekt des Programmteils, der die Züge anfordert. Trotzdem muss ein Spieler den Spielverlauf dokumentieren, damit er gültige Züge identifizieren kann. Dazu erzeugt jeder Spieler ein eigenes Spielbrett-Objekt und setzt seine und die Züge des Gegenspielers auf diesem Brett.

Daraus können sich Widersprüche zwischen dem Status des eigenen Spielbretts und dem gelieferten Status des Spielbretts des Hauptprogramms ergeben. Das ist ein Fehler auf den mit einer Exception reagiert wird.

\item Die Methoden der Player-Schnittstelle müssen in der richtigen Reihenfolge aufgerufen werden. Eine Abweichung davon ist ein Fehler auf den mit einer Exception reagiert werden muss.

Ein Spieler wird zu Spielbeginn mit einem Aufruf von \code{init} initialisiert und durchläuft danach die Methoden \code{request}, \code{confirm} und danach \code{update} bis das Spiel endet. Im Falle eines blauen Spielers beginnt der Spieler mit \code{update} statt \code{request}. Der Zeitpunkt des Spielbeginns und eines erneuten Spiels ist für den Spieler nicht ersichtlich, \code{init} kann zu einem beliebigen Zeitpunkt aufgerufen werden.

\newpage
\item Für ein problemloses Netzwerkspiel ist es nötig, dass die Spielerklassen nur \code{Exception}'s werfen und keine selbst erstellten Klassen, die von dieser erben. An jeder anderen Stelle im Spiel können eigene Exceptions frei erzeugt und geworfen werden.
\end{itemize}

Die Methoden dieser Schnittstelle sind wie folgt zu verstehen:

% Description player methods
\begin{itemize}[leftmargin=4em]
\item[\code{init}] \hfill \\Initialisiert den Spieler, sodass mit diesem Spieler-Objekt ein neues Spiel mit einem Spielbrett der Größe \code{size} $\times$ \code{size} und der durch den Parameter \code{color} bestimmten Farbe, begonnen werden kann.

Die Spielerfarbe ist einer der beiden Werte der Enumeration \\
\code{towerwarspp.preset.PlayerColor} und kann die Werte \code{RED} und \code{BLUE} haben.
\item[\code{request}] \hfill \\Fordert vom Spieler einen Zug an.
\item[\code{confirm}] \hfill \\Übergibt dem Spieler im Parameter \code{boardStatus} Informationen über den letzten mit \code{request} vom Spieler gelieferten Zug.

\underline{Beispiele}
\begin{itemize}
\item Gilt \code{boardStatus == }\textit{eigener Status} und\dots
\begin{itemize}
\item \dots \code{boardStatus == Status.OK} war der letzte Zug gültig
\item \dots \code{boardStatus == Status.RED\_WIN} war der letzte Zug gültig und der rote Spieler hat das Spiel gewonnen
\end{itemize}
\item Gilt \code{boardStatus != }\textit{eigener Status} wird eine Exception geworfen
\end{itemize}
\item[\code{update}] \hfill \\Liefert dem Spieler im Parameter \code{opponentMove} den letzten Zug des Gegners und im Parameter \code{boardStatus} Informationen über diesen Zug.

\underline{Hinweis}

Hier gelten die gleichen Beispiele wie auch für \code{confirm}.
\end{itemize}

% Human Player
\item Erstellen Sie eine Interaktive-Spieler-Klasse, die die Schnittstelle \\ \code{towerwarspp.preset.Player} implementiert.

Ein Interaktiver-Spieler benutzt ein Objekt einer Klasse, die das Interface \\ \code{towerwarspp.preset.Requestable} implementiert, um Züge vom Benutzer anzufordern.

% Random AI
\item Erstellen Sie eine Computerspieler-Klasse, die die Spieler-Schnittstelle implementiert und gültige, aber nicht notwendigerweise zielgerichtete, Züge generiert. Dazu wird aus allen aktuell möglichen gültigen Spielzügen zufällig ein Zug ausgewählt.

\underline{Hinweis}

Java stellt für die Erzeugung von Pseudozufallszahlen die Klasse \code{java.util.Random} zur Verfügung.

% Simple AI
\item Erstellen Sie einen weiteren Computerspieler, der zielgerichtet, entsprechend der unten beschriebenen \hyperlink{strategy}{einfachen Strategie}, versucht das Spiel zu gewinnen.

Verwenden Sie hierzu eine Instanz der erweiterten Spielbrett-Klasse, um die Züge nach dieser Strategie zu bewerten.

% Network
\item Programmieren Sie einen Netzwerkspieler mit dem sie jede Implementation der Schnittstelle \code{towerwarspp.preset.Player} einer anderen TowerWarsPP- \\Implementation anbieten können.

Falls Sie den Netzwerkspieler im Netzwerk anbieten möchten, läuft die Spiellogik auf einer entfernten TowerWarsPP-Implementation. Sehen Sie hierfür eine Möglichkeit vor, das Spiel dennoch über die selbst geschriebene Ausgabe-Schnittstelle zu verfolgen.

Im Anhang befinden sich Codebeispiele für den Umgang mit \code{RMI}.

\subsection*{Hauptprogramm}
% Run class
\item Erstellen Sie eine ausführbare Klasse mit folgender Funktionalität.
\begin{itemize}
\item Die Auswahl der roten und blauen Spieler Klassen (Interaktiver Spieler, einer der Computerspieler) und die Größe des Spielbretts soll beim Starten des Programms über die Kommandozeile festgelegt werden können.

Verwenden Sie zum Einlesen der Kommandozeilenparameter und zum Abfragen der entsprechenden Einstellungen ein Objekt der Klasse \\
\code{towerwarspp.preset.ArgumentParser}.

Im Anhang findet sich ein kleines kommentiertes Beispiel wie diese Klasse zu verwenden ist. Weitere Parameter dürfen bei Bedarf gerne hinzugefügt werden.
\item Ein Spielbrett in Ausgangsposition mit der eingestellten Größe wird initialisiert.
\item Zwei Spielerobjekte werden wie eingestellt erzeugt und über Referenzen der \code{towerwarspp.preset.Player}-Schnittstelle angesprochen. 

Beide Spieler benutzen dasselbe Objekt einer Klasse, die das \code{Requestable}-Interface implementiert, um Züge vom Benutzer anzufordern.
\item Von den Spieler-Referenzen werden abwechselnd Züge erfragt. Gültige Züge werden bestätigt und dem jeweils anderen Spieler mitgeteilt.
\item Die gültigen Züge werden auf dem Spielbrett ausgeführt.
\item Der aktuelle Stand des Spiels (und des Spielbretts) wird über die selbst geschriebene Ausgabe-Schnittstelle (textuell oder grafisch, je nach Konfiguration) ausgegeben.
\item \textbf{Wenn der Schalter \code{----debug} gesetzt ist, wird außerdem nach jedem Zug der ausgeführte Zug, das Spielbrett nach diesem Zug sowie der Status dieses Zuges auf der Standardausgabe ausgegeben.}

\underline{Hinweis}

\textbf{Die Debugausgabe darf von Ihnen um weitere Ausgaben ergänzt werden.} \marginpar{\textbf{|}}
\item Wenn ein Zug zum Spielende führt, macht die Ausgabe (textuell oder grafisch) eine Meldung darüber.
\item Sorgen Sie dafür, dass man das Spiel Computer gegen Computer gut verfolgen kann, verwenden Sie hierfür den Kommandozeilenparameter \code{-delay}.
\item Sehen Sie eine Möglichkeit vor über das Netzwerk zu spielen. \deleted{Mit den Schaltern \code{----local} und \code{----network} soll gesteuert werden ob es sich um ein lokales oder ein Netzwerkspiel handelt.} \marginpar{\textbf{|}}

\deleted{Sehen Sie im Falle eines Netzwerkspiels eine Möglichkeit vor, diese zu konfigurieren (Finden oder Anbieten eines Spielers, Art des Spielers der angeboten wird, Adresse:Port). Dies können Sie zum Beispiel mit Konfigurationsdateien oder weiteren Kommandozeilenparametern lösen.} \marginpar{\textbf{|}}

\textbf{Ein Netzwerkspiel findet statt, wenn mindestens einer der Spieler den Typ \code{REMOTE} hat (siehe \code{towerwarspp.preset.PlayerType}) oder wenn ein Spieler im Netzwerk angeboten wird.} \marginpar{\textbf{|}}

\textbf{Das Hauptspiel behandelt einen Netzwerkspieler über die Schnittstelle \code{towerwarspp.preset.Player} wie jeden anderen Spieler auch.} \marginpar{\textbf{|}}

\textbf{Sehen Sie im Falle eines \code{REMOTE}-Spielers eine Möglichkeit vor, diesen zu finden (Name, Host, Port). Dies können Sie zum Beispiel über weitere Kommandozeilenparameter steuern oder interaktiv abfragen.} \marginpar{\textbf{|}}

\textbf{Wenn Sie einen Netzwerkspieler anbieten möchten, wählen Sie auch hier eine geeignete Methode den Spielertypen und den Namen einzustellen, unter dem der Spieler an der RMI Registry registriert werden soll.} \marginpar{\textbf{|}}

\textbf{Beim Anbieten wird keine Farbe festgelegt, da der Spieler diese Information beim Aufruf von \code{init} mitgeteilt bekommt.} \marginpar{\textbf{|}}
\end{itemize}

% Ant
\item Verwenden Sie \emph{Ant} zum automatisierten Übersetzen des Programms und zum Erstellen der Dokumentation.

\item \textbf{Optional.} Bauen Sie das Spiel weiter aus.
\begin{itemize}
\item Laden und Speichern von Spielständen
\item Implementieren Sie einen Turniermodus
\item Erstellen Sie einen weiteren, intelligenteren Computerspieler, z.B. durch die Vorrausberechnung weiterer Züge und/oder einer besser balancierten und/oder erweiterten Bewertung.
\item \dots
\end{itemize}
\end{enumerate}
