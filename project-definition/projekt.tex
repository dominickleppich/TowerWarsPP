\documentclass[12pt]{article}

\usepackage[a4paper,top=1in, bottom=1in, left=1in, right=1in]{geometry}

\usepackage[utf8]{inputenc} %Eingabecodierung
\usepackage[ngerman]{babel} %Deutsche Sprache
\usepackage[T1]{fontenc} %T1 Zeichenkodierung, fuer automatische Zeilenumbrueche nach Umlauten

\usepackage{lmodern} %Moderner Zeichensatz
\usepackage{textcomp}

\usepackage{graphicx} %Grafiken laden
\usepackage{caption}

\usepackage{tikz,pgf} % TikZ Grafiken
\usepackage{pdfpages}
\usepackage{pgffor}
\usetikzlibrary{arrows}
\usetikzlibrary{calc}

\usepackage{enumitem} %Manuelle Numerierungsoptionen
\usepackage{float} % Figuren an der korrekten Quellcode Position anzeigen
\usepackage{pdflscape} % Querlegen

\usepackage{listings}
\lstset{
	basicstyle=\scriptsize\ttfamily,
	breakatwhitespace=false,
	commentstyle=\color{darkgray},
	frame=lines,
	framexleftmargin=1.5em,
	language=Java,
	numbers=left,
	numbersep=10pt,
	numberstyle=\tiny,
	rulecolor=\color{black},
	showstringspaces=true,
	tabsize=4,
	xleftmargin=1.5em
}

\usepackage{amsthm} %Mathematiksatz-Umgebung
\usepackage{amssymb} %Mathematische Symbole
\usepackage{amsmath}

\theoremstyle{plain}
\newtheorem{theorem}[equation]{Theorem}
\newtheorem{satz}[equation]{Satz}
\newtheorem{lemma}[equation]{Lemma}
\newtheorem{korollar}[equation]{Korollar}

\theoremstyle{definition}
\newtheorem{definition}[equation]{Definition}
\newtheorem{aufgabe}{Aufgabe}
\newtheorem*{behauptung}{Behauptung}
\newtheorem{uebung}{Übung}
\newtheorem*{loesung}{Lösung}

\theoremstyle{remark}
\newtheorem{beispiel}[equation]{Beispiel}
\newtheorem{hinweis}{Hinweis}

\newcommand{\ggT}{\textrm{ggT}}
\newcommand{\kgV}{\textrm{kgV}}
\newcommand{\code}[1]{\texttt{#1}} 

\setlength{\parindent}{0cm}
\setlength{\parskip}\medskipamount

\begin{document}

\textbf{Allgemeines Programmierpraktikum}\hfill \textbf{Sommersemester 2016} \\
Dr. Henrik Brosenne\hfill Georg-August-Universität Göttingen \\
\hspace*{1pt}\hfill Institut für Informatik
\bigskip
\hrule
\bigskip
\begin{center}
{\Large \textbf{Projekt} \\ \textbf{Vorankündigung}}
\end{center}
\bigskip
\textbf{Abgabe bis zum xx.xx.xxxx, xx:xx Uhr}. 

\textbf{Prüfungen in der Zeit vom xx.xx bis xx.xx.xxxx}. 

\section*{Organisation}
\begin{enumerate}
\item Wenn Sie an der Prüfung zum Modul \emph{B.Inf.1802: Programmierpraktikum} teilnehmen möchten, müssen Sie sich bis zum \textbf{xx.xx.xxxx} in \textbf{FlexNow} anmelden.
\item Bilden Sie Projektgruppen mit vier Teilnehmern, größere Gruppen müssen ausdrücklich genehmigt werden und bekommen zusätzliche Aufgaben.

%Sie bilden \textbf{nicht} automatisch mit den Teilnehmern Ihrer StudIP-Veranstaltung eine Projektgruppe.
\item Vereinbaren Sie mit einem Tutor, der Ihr Projekt betreuen soll, einen Termin für ein regelmäßiges Treffen.
\item Bestimmen Sie einen Projektleiter. Zu den Aufgaben des Projektleiters gehört es die Entwicklung des Projekts als Ganzes zu steuern.
\begin{itemize}
\item Werden alle Anforderungen erfüllt?
\item Sind Sprache und Stil der Dokumentation einheitlich?
\item Ist eine überarbeitete Klasse von einem weiteren Projektmitglied kontrolliert worden?
\item \dots
\end{itemize}
\item Geben Sie der Projektgruppe einen aussagefähigen Namen ungleich \code{breaktroughPP}.
\item Melden Sie Ihre Projektgruppe an, unter Angabe des Namens und der Teilnehmer 
(E-Mail an \code{brosenne@informatik.uni-goettingen.de}).
\item Wählen Sie für die Prüfung einen Termin im Testzeitraum und lassen Sie sich diesen Termin bestätigen (E-Mail an \code{brosenne@informatik.uni-goettingen.de}).
\end{enumerate} \newpage
\section*{Fusionforge}
\begin{enumerate}
\item Melden Sie sich unter \code{https://fusionforge.informatik.uni-goettingen.de/} als Benutzer an.
\item Bestimmen Sie einen aus Ihrer Gruppe zum Fusionforge-Administrator, der dann ein neues Fusionforge-Projekt, unter dem Namen der Projektgruppe, für Ihre Gruppe anlegt.

Folgende Einstellungen sind vorzunehmen.
\begin{enumerate}[label=\alph*)]
\item Als Source Code Management (SCM) wählen Sie Subversion (SVN).
\item Alle Gruppenmitglieder registrieren sich im Projekt (oder werden vom Administrator registriert).
\item Das SVN-Verzeichnis ist nicht öffentlich lesbar, also nur für Mitarbeiter des Projektes zugänglich.
\item Registrieren Sie auch Ihren Tutor als Projektmitarbeiter.
\end{enumerate}
\end{enumerate} \newpage
\section*{Prüfung}
Nach der Abgabe wird das Projekt als Ganzes bewertet.

Während der Prüfung stellt jeder Teilnehmer den Teil des Projektes vor, für dessen Implementation er verantwortlich ist. Besonders interessant sind die aufgetretenen Probleme und deren Lösungen.

Neben der korrekten Umsetzung der Projektanforderungen wird gut lesbar und strukturierter Quellcode erwartet. Es sollten die Grundlagen des objektorientierten Programmentwurfs (z.B. Kapselung, Vererbung, Polymorphismus) berücksichtigt und die Möglichkeiten von Java (z.B. \emph{Java Collections Framework} ausgenutzt werden.

Jedem Teilnehmer werden Fragen zum Projekt, sowie zu Java, JavaDoc, Subversion und Ant, im Rahmen von Vorlesung und Übung, gestellt. Daraus ergibt sich für jeden Teilnehmer eine Tendenz (z.B. etwas schwächer als das Projekt insgesamt) und letztlich eine individuelle Note.
\section*{Spiel}
Realisieren Sie \textbf{TowerWarsPP} als Computerspiel in Java. Kommentieren Sie den Quellcode ausführlich. Verwenden Sie JavaDoc für das \emph{Application Programming Interface (API)} und kommentieren Sie sonst wie üblich.

Die vorgegebenen Quelltexte finden Sie in der Stud.IP-Veranstaltung unter \\
\textit{Dateien\textrightarrow Projekt\textrightarrow TowerWarsPP.tgz}

Alle vorgegebenen Klassen sind im Anhang zu finden.

\begin{enumerate}
% Setup
\item Alle Klassen und Schnittstellen gehören zu einem Package, das mit \code{towerwarspp} beginnt.

Die vorgegebenen Klassen und Schnittstellen des Package \code{towerwarspp.preset} dürfen, mit Ausnahme von Kommentaren, nicht verändert werden. 

Es ist allerdings erlaubt folgende Klassen zu erweitern (nähere Informationen sind dem entsprechenden Abschnitt der Projektbeschreibung zu entnehmen):
\begin{description}
\item[\code{Viewer}] Darf komplett selbst geschrieben werden. Im Anhang befindet sich ein Viewer, der lediglich als Anregung dienen soll.
\item[\code{ArgumentParser}] Darf um weitere Parameter ergänzt werden.
\end{description}

\newpage
\subsection*{Spielbrett}

% Board
\item Erstellen Sie eine Spielbrett-Klasse mit folgenden Merkmalen:
\begin{itemize}
\item Ein Spielfeld mit $n \times n$ Feldern wird verwaltet. Dabei gilt $4 \le n \le 26$.
\item Implementieren Sie eine Methode, mit der Spielzüge entgegengenommen werden können.
\item Gültige Spielzüge und Spielzüge die zum Ende des Spiels führen werden erkannt.

Ein leerer Zug (\code{null}) ist gültig und wird als Aufgabe des Spielers gewertet.
\item Der erste entgegengenommene Spielzug gehört immer zum roten Spieler.
\item Ein Spielzug ist ein Objekt der Klasse \code{towerwarspp.preset.Move}, das mit Referenzen auf Objekte der Klasse \code{towerwarspp.preset.Position} arbeitet.

\item Implementieren Sie eine Methode, die einen Wert aus der Enumeration\\ \code{towerwarspp.preset.Status} zurückliefert, über die der Spielstand erfragt werden kann.

\item Die Schnittstelle \code{towerwarspp.preset.Viewable} wird implementiert.
\end{itemize}

% AI Board
\item Erstellen Sie eine erweiterte Spielbrett-Klasse, die von der vorherigen erbt und folgende Zusatzanforderungen erfüllt:

\begin{itemize}
\item Implementieren Sie eine Methode, die -- ausgehend von der aktuellen Spielsituation -- alle möglichen gültigen Spielzüge eines Spielers zurückliefert.

\item Implementieren Sie eine Methode, die Spielzüge anhand der unten beschriebenenen \hyperlink{strategy}{einfachen Strategie} bewertet.
\end{itemize}

\subsection*{Ein- und Ausgabe}

% Viewer
\item Erstellen Sie eine Klasse, die die Schnittstelle \code{towerwarspp.preset.Viewer} implementiert.

Diese Klasse soll es ermöglichen alle für das Anzeigen eines Spielbrett-Objekts nötigen Informationen zu erfragen, ohne Zugriff auf die Attribute des Spielbrett-Objekts zuzulassen.

Die Methode \code{viewer} des Spielbretts liefert ein passendes Objekt dieser Klasse.

% Text Input
\item Erstellen Sie eine Text-Eingabe-Klasse, die die Schnittstelle \\ \code{towerwarspp.preset.Requestable} implementiert.

Die Methode \code{deliver} fordert einen Zug, in einer Zeile, von der Standardeingabe an und liefert ein dazu passendes \code{towerwarspp.preset.Move}-Objekt zurück.

Verwenden Sie die statische Methode \code{parseMove} der Klasse \code{Move} um den von der Standardeingabe eingelesenen String in ein Move-Objekt umzuwandeln.

Die Methode \code{parseMove} wirft eine \code{towerwarspp.preset.MoveFormatException}, falls das Einlesen missglücken sollte. Auf diese Exception muss sinnvoll reagiert werden.

% Text Output
\item Erstellen Sie eine Text-Ausgabe-Klasse mit der der aktuelle Spielstand eines Spielbretts auf die Standardausgabe geschrieben werden kann. Dazu wird ein Objekt einer Klasse, die die Schnittstelle \code{towerwarspp.preset.Viewer} implementiert, benutzt.

Sehen Sie für die Ausgabe eine Schnittstelle vor die sich an \code{Requestable} orientiert und implementieren Sie dieses in der Text-Ausgabe-Klasse.

\underline{Hinweis}

Text-Eingabe- und Ausgabe-Klasse können in einer Klasse vereint werden.

% Graphic
\item Erstellen Sie eine grafische Ein-Ausgabe-Klasse. Diese Klasse implementiert die Schnittstellen \code{towerwarspp.preset.Requestable} und die von Ihnen geschriebene Ausgab-eSchnittstelle und benutzt ein Objekt einer Klasse, die die Schnittstelle \code{towerwarspp.preset.Viewer} implementiert, für eine einfache grafische Ausgabe.

Sorgen Sie dafür, dass die Grafik mit ändernder Fenstergröße sinnvoll skaliert.

\underline{Hinweis}

Investieren Sie nicht zu viel Zeit in das Design, denn es wird nicht bewertet.

Sie können nützliche Informationen zum Zeichnen und Arbeiten mit hexagonalen Spielfeldern auf dieser Seite finden: \href{http://www.redblobgames.com/grids/hexagons}{http://www.redblobgames.com/grids/hexagons}

\subsection*{Spieler}

% Player
\item Alle Spieler implementieren die Schnittstelle \code{towerwarspp.preset.Player}.

\begin{itemize}
\item Ein Spieler hat keine Referenz auf das Spielbrett-Objekt des Programmteils, der die Züge anfordert. Trotzdem muss ein Spieler den Spielverlauf dokumentieren, damit er gültige Züge identifizieren kann. Dazu erzeugt jeder Spieler ein eigenes Spielbrett-Objekt und setzt seine und die Züge des Gegenspielers auf diesem Brett.

Daraus können sich Widersprüche zwischen dem Status des eigenen Spielbretts und dem gelieferten Status des Spielbretts des Hauptprogramms ergeben. Das ist ein Fehler auf den mit einer Exception reagiert wird.

\item Die Methoden der Player-Schnittstelle müssen in der richtigen Reihenfolge aufgerufen werden. Eine Abweichung davon ist ein Fehler auf den mit einer Exception reagiert werden muss.

Ein Spieler wird zu Spielbeginn mit einem Aufruf von \code{init} initialisiert und durchläuft danach die Methoden \code{request}, \code{confirm} und danach \code{update} bis das Spiel endet. Im Falle eines blauen Spielers beginnt der Spieler mit \code{update} statt \code{request}. Der Zeitpunkt des Spielbeginns und eines erneuten Spiels ist für den Spieler nicht ersichtlich, \code{init} kann zu einem beliebigen Zeitpunkt aufgerufen werden.


\item Für ein problemloses Netzwerkspiel ist es nötig, dass die Spielerklassen nur \code{Exception}'s werfen und keine selbst erstellten Klassen, die von dieser erben. An jeder anderen Stelle im Spiel können eigene Exceptions frei erzeugt und geworfen werden.
\end{itemize}

\newpage
Die Methoden dieser Schnittstelle sind wie folgt zu verstehen:

% Description player methods
\begin{itemize}[leftmargin=4em]
\item[\code{init}] \hfill \\Initialisiert den Spieler, sodass mit diesem Spieler-Objekt ein neues Spiel mit einem Spielbrett der Größe \code{size} $\times$ \code{size} und der durch den Parameter \code{color} bestimmten Farbe, begonnen werden kann.

Die Spielerfarbe ist einer der beiden Werte der Enumeration \\
\code{towerwarspp.preset.PlayerColor} und kann die Werte \code{RED} und \code{BLUE} haben.
\item[\code{request}] \hfill \\Fordert vom Spieler einen Zug an.
\item[\code{confirm}] \hfill \\Übergibt dem Spieler im Parameter \code{boardStatus} Informationen über den letzten mit \code{request} vom Spieler gelieferten Zug.

\underline{Beispiele}
\begin{itemize}
\item Gilt \code{boardStatus == }\textit{eigener Status} und\dots
\begin{itemize}
\item \dots \code{boardStatus == Status.OK} war der letzte Zug gültig
\item \dots \code{boardStatus == Status.RED\_WIN} war der letzte Zug gültig und der rote Spieler hat das Spiel gewonnen
\end{itemize}
\item Gilt \code{boardStatus != }\textit{eigener Status} wird eine Exception geworfen
\end{itemize}
\item[\code{update}] \hfill \\Liefert dem Spieler im Parameter \code{opponentMove} den letzten Zug des Gegners und im Parameter \code{boardStatus} Informationen über diesen Zug.

\underline{Hinweis}

Hier gelten die gleichen Beispiele wie auch für \code{confirm}.
\end{itemize}

% Human Player
\item Erstellen Sie eine Interaktive-Spieler-Klasse, die die Schnittstelle \\ \code{towerwarspp.preset.Player} implementiert.

Ein Interaktiver-Spieler benutzt ein Objekt einer Klasse, die das Interface \\ \code{towerwarspp.preset.Requestable} implementiert, um Züge vom Benutzer anzufordern.

% Random AI
\item Erstellen Sie eine Computerspieler-Klasse, die die Spieler-Schnittstelle implementiert und gültige, aber nicht notwendigerweise zielgerichtete, Züge generiert. Dazu wird aus allen aktuell möglichen gültigen Spielzügen zufällig ein Zug ausgewählt.

\underline{Hinweis}

Java stellt für die Erzeugung von Pseudozufallszahlen die Klasse \code{java.util.Random} zur Verfügung.

% Simple AI
\item Erstellen Sie einen weiteren Computerspieler, der zielgerichtet, entsprechend der unten beschriebenen \hyperlink{strategy}{einfachen Strategie}, versucht das Spiel zu gewinnen.

Verwenden Sie hierzu eine Instanz der erweiterten Spielbrett-Klasse, um die Züge nach dieser Strategie zu bewerten.

% Network
\item Programmieren Sie einen Netzwerkspieler mit dem sie jede Implementation der Schnittstelle \code{towerwarspp.preset.Player} einer anderen TowerWarsPP- \\Implementation anbieten können.

Falls Sie den Netzwerkspieler im Netzwerk anbieten möchten, läuft die Spiellogik auf einer entfernten TowerWarsPP-Implementation. Sehen Sie hierfür eine Möglichkeit vor, das Spiel dennoch über die selbst geschriebene Ausgabe-Schnittstelle zu verfolgen.

Im Anhang befinden sich Codebeispiele für den Umgang mit \code{RMI}.

\subsection*{Hauptprogramm}
% Run class
\item Erstellen Sie eine ausführbare Klasse mit folgender Funktionalität.
\begin{itemize}
\item Die Auswahl der roten und blauen Spieler Klassen (Interaktiver Spieler, einer der Computerspieler) und die Größe des Spielbretts soll beim Starten des Programms über die Kommandozeile festgelegt werden können.

Verwenden Sie zum Einlesen der Kommandozeilenparameter und zum Abfragen der entsprechenden Einstellungen ein Objekt der Klasse \\
\code{towerwarspp.preset.ArgumentParser}.

Im Anhang findet sich ein kleines kommentiertes Beispiel wie diese Klasse zu verwenden ist. Weitere Parameter dürfen bei Bedarf gerne hinzugefügt werden.
\item Ein Spielbrett in Ausgangsposition mit der eingestellten Größe wird initialisiert.
\item Zwei Spielerobjekte werden wie eingestellt erzeugt und über Referenzen der \code{towerwarspp.preset.Player}-Schnittstelle angesprochen. 

Beide Spieler benutzen dasselbe Objekt einer Klasse, die das \code{Requestable}-Interface implementiert, um Züge vom Benutzer anzufordern.
\item Von den Spieler-Referenzen werden abwechselnd Züge erfragt. Gültige Züge werden bestätigt und dem jeweils anderen Spieler mitgeteilt.
\item Die gültigen Züge werden auf dem Spielbrett ausgeführt.
\item Der aktuelle Stand des Spiels (und des Spielbretts) wird über die selbst geschriebene Ausgabe-Schnittstelle (textuell oder grafisch, je nach Konfiguration) ausgegeben.
\item Wenn ein Zug zum Spielende führt, macht die Ausgabe (textuell oder grafisch) eine Meldung darüber.
\item Sorgen Sie dafür, dass man das Spiel Computer gegen Computer gut verfolgen kann, verwenden Sie hierfür den Kommandozeilenparameter \code{-delay}.
\item Sehen Sie eine Möglichkeit vor über das Netzwerk zu spielen. Mit den Schaltern \code{----local} und \code{----network} soll gesteuert werden ob es sich um ein lokales oder ein Netzwerkspiel handelt. 

Sehen Sie im Falle eines Netzwerkspiels eine Möglichkeit vor, diese zu konfigurieren (Finden oder Anbieten eines Spielers, Art des Spielers der angeboten wird, Adresse:Port). Dies können Sie zum Beispiel mit Konfigurationsdateien oder weiteren Kommandozeilenparametern lösen.
\end{itemize}

% Ant
\item Verwenden Sie \emph{Ant} zum automatisierten Übersetzen des Programms und zum Erstellen der Dokumentation.

\item \textbf{Optional.} Bauen Sie das Spiel weiter aus.
\begin{itemize}
\item Laden und Speichern von Spielständen
\item Implementieren Sie einen Turniermodus
\item Erstellen Sie einen weiteren, intelligenteren Computerspieler, z.B. durch die Vorrausberechnung weiterer Züge und/oder einer besser balancierten und/oder erweiterten Bewertung.
\item \dots
\end{itemize}
\end{enumerate}
 \newpage
%\section*{KI-Wettbewerb}
Wenn mindestens zwei Gruppen Interesse daran haben ihre Computerspieler gegeneinander antreten zu lassen wird ein KI-Wettbewerb stattfinden.

Zur Teilnahme am KI-Wettbewerb muss ein Netzwerkspieler implementiert sein.

Abhängig von der Anzahl der Teilnehmer wird ein entsprechender Spielplan erstellt. Dieser legt dann fest, welches Team auf einem Brett welcher Größe mit welcher Farbe gegen welches andere Team antritt.

Ein Matchmaking-Programm wird alle im Netzwerk zur Verfügung gestellten Spieler-Objekte verwalten und dann anhand des Spielplans mehrere Spiele starten. Vor jedem Spiel wird bei den Spieler-Objekten die Methode \code{init} mit den entsprechenden Parametern aufgerufen. Stellen Sie daher sicher, dass Ihre Spieler-Klassen beliebig oft mit neuen Werten initialisiert werden können.

Die genaue Erzeugung des Spielplans steht noch nicht fest, aber es kann davon ausgegangen werden, dass zwei konkurrierende Teams mindestens zwei Spiele gegeneinander (je eines mit den roten und blauen Steinen) spielen werden.

Es gewinnt das Team, dessen KI die meisten Spiele gewonnen hat. Im Falle eines Unentschiedens wird am Tage des Wettbewerbs vor Ort eine Lösung ausgehandelt.

Um fair zu bleiben und stundenlanges Rechnen zu unterbinden werden die folgenden Regeln gelten.
\begin{itemize}
\item Verursacht ein Spieler eine Exception oder reagiert dieser mit inkorrektem Verhalten (z.B. bei Aufruf der Methoden \code{request}, \code{confirm}, \code{update} und \code{init}) gilt dieses Spiel für den verursachenden Spieler als Niederlage.
\item Benötigt ein Spieler zu viel Zeit für einen einzelnen Zug, wird das bei mehrfacher Wiederholung als Aufgabe dieses Spielers gewertet und der andere Spieler gewinnt. Ein genaues Zeitlimit wird rechtzeitig bekannt gegeben aber eine Rechenzeit von 5 Sekunden wird auf jeden Fall möglich sein.
\item Überschreitet ein Spieler ein Limit an Gesamtzeit, wird auch das als Aufgabe gewertet.
\end{itemize}

Sie können Ihre Implementation gegen eine fertige BreakthroughPP-Implemenation testen, um zu prüfen ob Ihre KI funktioniert und ob Zeitlimits eingehalten werden. Nähere Hinweise dazu werden bekannt gegeben.

Bei Fragen oder Problemen zum KI-Wettbewerb können Sie sich gerne per E-Mail an \\
\code{dominick.leppich@stud.uni-goettingen.de} wenden. \newpage
\section*{TowerWarsPP}
\newcommand{\myAlph}[1]{\char\numexpr`A-1+#1\relax}
\newcommand{\myalph}[1]{\char\numexpr`a-1+#1\relax}

\subsection*{Allgemein}
Das Spiel TowerWarsPP ist ein Spiel für zwei Spieler, wobei ein Spieler mit roten und der andere mit blauen Steinen spielt.

TowerWarsPP wird auf einem Spielfeld mit hexagonalen Feldern gespielt, die in einem Parallelogramm mit $n \times n$ Feldern angeordnet sind. Die Zeilen werden durchnummeriert, die Spalten werden mit Buchstaben bezeichnet. Die obere linke Ecke des Spielfeldes ist mit $A1$ gekennzeichnet.

\subsection*{Startaufstellung}
Für $n$ gilt $4 \le n \le 26$ und für die Distanz $d$ gilt 
\[d = \left\lfloor \frac{n}{2}\right\rfloor\hspace{10pt}.\] 
Das Feld $A1$ steht die rote Basis, auf dem Feld $Xn$ steht die blaue Basis (wobei das $X$ für den $n$-ten Buchstaben steht). 

Alle Felder die einen Abstand $d' \le d$ zur eigenen Basis haben, werden mit Steinen des jeweiligen Spielers befüllt. Der Abstand ist hierbei so definiert, dass Felder, die sich über eine Seite berühren, den Abstand $1$ zueinander haben.

\underline{Beispiel}

TowerWarsPP-Spielfeld aus $8 \times 8$ Feldern mit Startaufstellung.
\begin{center}
\includegraphics[scale=0.25]{graphic/start8.png}
\end{center}

Ein Nachbar eines Feldes ist ein Feld mit Abstand $1$ zu diesem. Jedes Feld in der Mitte hat demnach $6$ Nachbarn.

\subsection*{Spielablauf}
Der Spieler mit den roten Steinen hat den ersten Zug. Dann wird abwechselnd gezogen, einen Zug auszulassen ist nicht möglich. Ein Feld in TowerWarsPP ist entweder leer oder enthält eine Basis, einen Stein oder einen Turm. In jedem Zug muss ein Spieler genau einen Stein von einem auf ein anderes Feld bewegen.

\subsubsection*{Basis}
Jeder Spieler hat zu Beginn des Spiels eine Basis, deren Position festgelegt ist. Diese kann nicht bewegt werden. Es ist auch nicht möglich eigene Figuren auf die eigene Basis zu bewegen.

Es ist Ziel des Spiels die gegnerische Basis zu zerstören und folglich auch die eigene zu verteidigen.

\subsubsection*{Steine}
Jeder Spieler hat zu Beginn eine vordefinierte Menge an Steinen. 

Ein Stein kann sich von seiner Position aus ein Feld in jede Richtung bewegen (außer auf die eigene Basis natürlich).
\begin{figure}[ht]
\begin{center}
\includegraphics[scale=0.25]{graphic/token-nobase.png} \\
\smallskip
{\footnotesize $B1$ kann nicht auf Basis $A1$ ziehen}
\end{center}
\end{figure}

Zieht ein Stein auf ein leeres Feld, ist das ein \emph{Nahzug}. Zieht ein Stein auf ein Feld mit einem gegnerischen Stein, ist dieser geschlagen und der eigene Stein nimmt dessen Platz ein.
\begin{figure}[ht]
\begin{center}
\includegraphics[scale=0.25]{graphic/token-move-kick.png} \\
\smallskip
{\footnotesize $I7$ kann auf ein benachbartes leeres Feld ziehen oder $J6$ schlagen}
\end{center}
\end{figure}

Ein Stein kann nicht auf die eigene Basis ziehen, sehr wohl jedoch auf die gegnerische Basis (was zum Sieg führt).

Zieht ein Stein auf ein Feld mit einem eigenen Stein, so wird auf diesem Feld ein Turm der Höhe $1$ gebaut.

\subsubsection*{Türme}
Zu Beginn hat ein Spieler keine Türme. Türme können durch Ziehen eigener Steine auf eigene Steine gebaut und durch Ziehen eigener Steine auf Türme erhöht werden. Es ist durchaus möglich ein Spiel ohne Bauen eines einzigen Turms zu spielen und zu gewinnen.

Türme haben eine Höhe $h \le h_\text{max}$, wobei \[h_\text{max} = \left\lfloor \frac{n}{3}\right\rfloor\]

Zieht ein Stein auf ein Feld mit einem eigenen Turm, so wird dessen Höhe um $1$ erhöht.

Beim Erhöhen eines Turms darf die Maximalhöhe nie überschritten werden.

\bigskip

Türme erhöhen die Reichweite angrenzender Steine um die Höhe des Turmes, dabei spielt es keine Rolle ob der Stein in dieser Reichweite Ziehen oder Schlagen will.

Hat ein Stein einen benachbarten Turm, so erhöht sich die Reichweite des Steins um die Höhe $h$ des Turmes. Der Stein kann dann auf alle erlaubten Felder mit Abstand $d \le 1 + h$ ziehen. Hat ein Stein mehrere angrenzende Türme, so summieren sich deren Höhen bei der Bestimmung der möglichen Zugreichweite auf. Ein Stein mit $6$ benachbarten Türmen kann folglich alle Felder mit folgendem Abstand erreichen \[d \le 1 + \sum_{i=1}^6 h_i\hspace*{10pt}.\]

Zieht ein Stein auf ein benachbartes Feld spricht man von einem \emph{Nahzug}, sonst von einem \emph{Fernzug}.

\begin{figure}[ht]
\begin{center}
\includegraphics[scale=0.25]{graphic/move-ranged.png} \\
\smallskip
{\footnotesize $I7$ hat durch den Turm auf $J6$ die Reichweite $2$}
\end{center}
\end{figure}
\newpage
\begin{figure}[ht]
\begin{center}
\includegraphics[scale=0.25]{graphic/move-rangeddouble.png} \\
\smallskip
{\footnotesize $I7$ hat durch die Türme auf $J6$ und $H8$ die Reichweite $3$}
\end{center}
\end{figure}

\bigskip

Auch Türme haben Zugmöglichkeiten. Es ist nicht möglich Türme in Ihrer Gesamtheit zu bewegen. 

Allerdings kann ein Turm abgebaut werden, indem der oberste Stein vom Turm auf ein benachbartes Feld gezogen werden. Angrenzende Türme verleihen keinen Bonus auf die Reichweite beim Turmabbau. Es kann nur auf ein leeres oder eigenes Feld gezogen werden, ohne dabei die Maximalhöhe von Türmen zu überschreiten.

Beim Abbau des Turmes wird dessen Höhe natürlich um $1$ reduziert. Wird ein Turm der Höhe $1$ abgebaut, so bleibt dort ein normaler Stein zurück.

Bei diesem Abbau eines Turmes ist es nicht erlaubt zu schlagen. Daraus ergibt sich direkt, dass Türme nicht die gegnerische Basis schlagen können. Weiter ist es möglich, dass ein Spieler hierdurch keine Handlungsmöglichkeit mehr hat, wenn er nur noch einen Turm besitzt, der komplett von benachbarten gegnerischen Figuren besetzt ist. 

\begin{figure}[ht]
\begin{center}
\includegraphics[scale=0.25]{graphic/tower-move.png} \\
\smallskip {\footnotesize
\begin{itemize}
\item Turm $I7$ kann auf ein leeres Feld abgebaut werden
\item Turm $I7$ kann auf den eigenen Stein $H8$ abgebaut werden und damit einen neuen Turm bauen
\item Turm $I7$ kann auf den eigenen Turm $H7$ abgebaut werden und diesen damit erhöhen
\item Turm $I7$ darf weder nach $J6$ noch nach $J7$ abgebaut werden, da dort gegnerische Figuren stehen
\end{itemize}}
\end{center}
\end{figure}

\begin{figure}[ht]
\begin{center}
\includegraphics[scale=0.25]{graphic/neighbor-blocked-tower.png} \\
\smallskip
{\footnotesize Turm $I7$ hat keine Zugmöglichkeiten mehr}
\end{center}
\end{figure}

\newpage 
Gegnerische Türme können von Steinen geschlagen und \emph{blockiert} werden. 

Zieht ein Stein auf ein Feld mit einem gegnerischen Turm entscheidet die Art des Zuges was passiert. Handelt es sich um einen Nahzug, wird der gesamte gegnerische Turm geschlagen. Im Falle eines Fernzuges wird der gegnerische Turm blockiert.

\begin{figure}[ht]
\begin{center}
\includegraphics[scale=0.25]{graphic/token-kick-block-tower.png} \\
\smallskip
{\footnotesize $I6$ kann gegnerischen Turm $J6$ schlagen, \\$H7$ kann den gegnerischen Turm blockieren}
\end{center}
\end{figure}

Ist ein Turm blockiert, kann dieser weder abgebaut werden noch gibt er eigenen Steinen einen Bonus auf die Zugreichweite. Er steht nur noch blockiert auf dem Feld. 

\begin{figure}[ht]
\begin{center}
\includegraphics[scale=0.25]{graphic/range-blockedbonus.png} \\
\smallskip
{\footnotesize $I7$ hat einen Bonus durch Turm $J6$ aber nicht durch blockierten Turm $I6$}
\end{center}
\end{figure}

Diese Blockade kann wieder aufgehoben werden, indem ein eigener Steinauf den blockierten Turm zieht. Dies kann durch einen Nah- oder Fernzug oder durch Abbau eines angrenzenden Turms auf diesen Turm erfolgen.

Ein blockierter Turm kann nicht erneut blockiert werden. Beim Blockieren eines Turms wird die Höhe nicht verringert, beim Aufheben der Blockade nicht erhöht.

\newpage

\begin{figure}[ht]
\begin{center}
\includegraphics[scale=0.25]{graphic/tower-blocked.png} \\
\smallskip
{\footnotesize Turm $I7$ ist blockiert und kann nicht abgebaut werden}
\end{center}
\end{figure}

\begin{figure}[ht]
\begin{center}
\includegraphics[scale=0.25]{graphic/tower-unblock.png} \\
\smallskip
{\footnotesize Die Blockade von Turm $I7$ kann durch Abbauen des Turms $I6$\\ oder durch den Stein auf $J6$ aufgehoben werden}
\end{center}
\end{figure}

\subsection*{Ende des Spiels}
Das Spiel ist gewonnen, wenn eine der folgenden Situationen eintritt:
\begin{itemize}
\item Ein Spieler zieht einen Stein auf die gegnerische Basis
\item Ein Spieler schlägt alle Steine des Gegenspielers
\item Ein Spieler macht den anderen Spieler handlungsunfähig
\item Der Gegenspieler gibt auf
\end{itemize}

Da in jedem Zug ein Stein bewegt werden muss und im Falle von Handlungsunfähigkeit der Gegner gewinnt, kann das Spiel nicht unentschieden enden. Es ist jedoch sehr gut möglich, dass ein Spiel endlos andauert.

\subsection*{Herkunft von TowerWars}
Die Spielidee für TowerWarsPP wurde im Verlaufe dieses Semesters von Ole Umlauft und Dominick Leppich entwickelt und im Laufe von Testspielen zu dem ausgebaut, was jetzt in dieser Spielbeschreibung definiert ist.

Die Spielidee ist unter der \emph{Creative-Commons}-Lizenz ``CC BY-NC'' veröffentlicht.

Wir würden uns freuen über Modifikationen des Spiels informiert zu werden.

\href{mailto:dominick.leppich@gmail.com}{dominick.leppich@gmail.com} \\
\href{mailto:ole.umlauft@gmail.com}{ole.umlauft@gmail.com}
\begin{flushright}
\includegraphics[scale=0.7]{graphic/cc.png}
\end{flushright} \newpage
\hypertarget{strategy}{\section*{Einfache Strategie}}
\underline{In Richtung gegnerischer Basis ziehen}

Es ist eine gute Idee in Richtung der gegnerischen Basis zu ziehen und weniger gut sich von ihr wegzubewegen. Für jeden Zug sei $d_1$ der Abstand zur gegnerischen Basis vor dem Zug und $d_2$ der Abstand nach dem Zug. Daraus berechnet sich ein Basis-Abstandsbonus \[r_\text{Basis} = d_1 - d_2\]

\underline{Steine und Türme schlagen}

Weiterhin ist es klug gegnerische Einheiten zu schlagen. Berechne hierfür einen Bonus für das Schlagen von gegnerischen Einheiten \[b_\text{kick} = \begin{cases}
0, & \text{wenn keine Einheit geschlagen wurde} \\
1, & \text{wenn ein Stein geschlagen wurde} \\
2, & \text{wenn ein Turm geschlagen wurde} \\
1337, & \text{wenn die gegnerische Basis geschlagen wurde}
\end{cases}\]

\underline{Gesamtwertung}

Berechnen Sie daraus für jeden Zug eine Bewertung
\[score = r_\text{Basis} + b_\text{kick}\]

und wählen Sie unter den Zügen mit Maximalbewertung zufällig einen aus.

\bigskip

Zusätzlich gelten die beiden folgenden Regeln.
\begin{enumerate}[label=\alph*)]
\item Gibt es einen Zug mit dem das Spiel gewonnen werden kann, dann wird dieser ausgeführt.
\item Ein Zug, der zur Folge hätte, dass der Gegner im nächsten Zug gewinnen könnte, wird dieser nur ausgeführt, wenn es keine Alternative gibt.
\end{enumerate} \newpage
\section*{Quellcode}
\subsection*{Preset}
Die Klasse \code{Position}, die Positionen auf dem Spielbrett beschreibt.
\lstinputlisting{files/preset/Position.java}

Die Klasse \code{Move}, die Spielzüge beschreibt.
\lstinputlisting{files/preset/Move.java}

Exceptions die beim Erzeugen von Positionen und Zügen aus Strings geworfen werden können.
\lstinputlisting{files/preset/PositionFormatException.java}
\lstinputlisting{files/preset/MoveFormatException.java} \newpage

Interface für den Spieler und Enumerations für Spielerfarbe, Spielertyp und Status.
\lstinputlisting{files/preset/Player.java}
\lstinputlisting{files/preset/PlayerColor.java}
\lstinputlisting{files/preset/PlayerType.java}
\lstinputlisting{files/preset/Status.java}

Die Interfaces zum Kapseln der Ein- und Ausgabe.
\lstinputlisting{files/preset/Requestable.java}
\lstinputlisting{files/preset/Viewable.java}
\lstinputlisting{files/preset/Viewer.java}

\newpage
\subsection*{Argumentparser}
Der ArgumentParser mit seiner Exception.
\lstinputlisting{files/preset/ArgumentParser.java}
\lstinputlisting{files/preset/ArgumentParserException.java}

\newpage
Ein Beispiel zur Verwendung des Parsers.
\begin{lstlisting}
package towerwarspp.demo;

import towerwarspp.preset.ArgumentParser;
import towerwarspp.preset.ArgumentParserException;

public class ArgumentParserTest {
    public static void main(String[] args) {
        try {
            ArgumentParser ap = new ArgumentParser(args);

            System.out.println("size: " + ap.getSize());

            switch (ap.getRed()) {
                case HUMAN:
                    System.out.println("Red chose human player");
                    break;
                case RANDOM_AI:
                    System.out.println("Red chose random player");
                    break;

                // and so on..
            }
        } catch (ArgumentParserException e) {
            // Something went wrong...
            e.printStackTrace();
        }
    }
}

\end{lstlisting}

\newpage
\subsection*{Netzwerk (RMI)}
Ein Gerüst für einen Netzwerkspieler finden Sie nachfolgend.
\begin{lstlisting}
package towerwarspp.demo;

import java.rmi.*;
import java.rmi.server.UnicastRemoteObject;
import java.net.*;

import towerwarspp.preset.*;

public class NetPlayer extends UnicastRemoteObject implements Player {
	private Player player;
	
	// ---------------------------------------------
	
	public NetPlayer(Player player) throws RemoteException {
		this.player = player;
	}
	
	public Move request() throws Exception, RemoteException {
		return player.request();
	}
	
	public void update(Move opponentMove, Status boardStatus)
		throws Exception, RemoteException {
		player.update(opponentMove, boardStatus);
	}
	
	public void confirm(Status boardStatus)
		throws Exception, RemoteException {
		player.confirm(boardStatus);
	}		
	
	public void init(int size, PlayerColor color)
		throws Exception, RemoteException {
		player.init(size, color);	
	} 
}
\end{lstlisting}

Die folgende Methode \code{offer} ist ein Beispiel wie ein Netzwerkspieler an einen Rechner \code{host}, auf dem eine RMI-Registry läuft, unter einem beliebigen Namen \code{name} angeboten werden kann.
\begin{lstlisting}
	public void offer(Player p, String host, String name) {
		try {
			Naming.rebind("rmi://" + host + "/" + name, p);
			System.out.println("Player (" + name + ") ready");
		} catch (MalformedURLException ex) {
			ex.printStackTrace();
		} catch (RemoteException ex) {
			ex.printStackTrace();
		}
	}
\end{lstlisting}

Die folgende Methode \code{find} ist ein Beispiel wie ein Spieler auf einen Rechner \code{host}, auf dem eine RMI-Registry läuft, unter einem bekannten Namen \code{name} gefunden werden kann.
\begin{lstlisting}
	public Player find(String host, String name) {
		Player p = null;
		try {
			p = (Player) Naming.lookup("rmi://" + host + "/" + name);
			System.out.println("Player (" + name + ") found");
		} catch (Exception ex) {
			ex.printStackTrace();
		}
		return p;
	}
\end{lstlisting}


\end{document}