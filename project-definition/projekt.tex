\documentclass[12pt]{article}

\usepackage[a4paper,top=1in, bottom=1in, left=1in, right=1in]{geometry}

\usepackage[utf8]{inputenc} %Eingabecodierung
\usepackage[ngerman]{babel} %Deutsche Sprache
\usepackage[T1]{fontenc} %T1 Zeichenkodierung, fuer automatische Zeilenumbrueche nach Umlauten

\usepackage{lmodern} %Moderner Zeichensatz
\usepackage{textcomp}

\usepackage{graphicx} %Grafiken laden

\usepackage{tikz,pgf} % TikZ Grafiken
\usepackage{pgffor}
\usetikzlibrary{arrows}
\usetikzlibrary{calc}

\usepackage{enumitem} %Manuelle Numerierungsoptionen
\usepackage{float} % Figuren an der korrekten Quellcode Position anzeigen
\usepackage{pdflscape} % Querlegen

\usepackage{listings}
\lstset{
	basicstyle=\scriptsize\ttfamily,
	breakatwhitespace=false,
	commentstyle=\color{darkgray},
	frame=lines,
	framexleftmargin=1.5em,
	language=Java,
	numbers=left,
	numbersep=10pt,
	numberstyle=\tiny,
	rulecolor=\color{black},
	showstringspaces=true,
	tabsize=4,
	xleftmargin=1.5em
}

\usepackage{amsthm} %Mathematiksatz-Umgebung
\usepackage{amssymb} %Mathematische Symbole
\usepackage{amsmath}

\theoremstyle{plain}
\newtheorem{theorem}[equation]{Theorem}
\newtheorem{satz}[equation]{Satz}
\newtheorem{lemma}[equation]{Lemma}
\newtheorem{korollar}[equation]{Korollar}

\theoremstyle{definition}
\newtheorem{definition}[equation]{Definition}
\newtheorem{aufgabe}{Aufgabe}
\newtheorem*{behauptung}{Behauptung}
\newtheorem{uebung}{Übung}
\newtheorem*{loesung}{Lösung}

\theoremstyle{remark}
\newtheorem{beispiel}[equation]{Beispiel}
\newtheorem{hinweis}{Hinweis}

\newcommand{\ggT}{\textrm{ggT}}
\newcommand{\kgV}{\textrm{kgV}}
\newcommand{\code}[1]{\texttt{#1}} 

\setlength{\parindent}{0cm}
\setlength{\parskip}\medskipamount

\begin{document}

\textbf{Allgemeines Programmierpraktikum}\hfill \textbf{Sommersemester 2016} \\
Dr. Henrik Brosenne\hfill Georg-August-Universität Göttingen \\
\hspace*{1pt}\hfill Institut für Informatik
\bigskip
\hrule
\bigskip
\begin{center}
{\Large \textbf{Projekt} \\ \textbf{Vorankündigung}}
\end{center}
\bigskip
\textbf{Abgabe bis zum xx.xx.xxxx, xx:xx Uhr}. 

\textbf{Prüfungen in der Zeit vom xx.xx bis xx.xx.xxxx}. 

\section*{Organisation}
\begin{enumerate}
\item Wenn Sie an der Prüfung zum Modul \emph{B.Inf.1802: Programmierpraktikum} teilnehmen möchten, müssen Sie sich bis zum \textbf{xx.xx.xxxx} in \textbf{FlexNow} anmelden.
\item Bilden Sie Projektgruppen mit vier Teilnehmern, größere Gruppen müssen ausdrücklich genehmigt werden und bekommen zusätzliche Aufgaben.

%Sie bilden \textbf{nicht} automatisch mit den Teilnehmern Ihrer StudIP-Veranstaltung eine Projektgruppe.
\item Vereinbaren Sie mit einem Tutor, der Ihr Projekt betreuen soll, einen Termin für ein regelmäßiges Treffen.
\item Bestimmen Sie einen Projektleiter. Zu den Aufgaben des Projektleiters gehört es die Entwicklung des Projekts als Ganzes zu steuern.
\begin{itemize}
\item Werden alle Anforderungen erfüllt?
\item Sind Sprache und Stil der Dokumentation einheitlich?
\item Ist eine überarbeitete Klasse von einem weiteren Projektmitglied kontrolliert worden?
\item \dots
\end{itemize}
\item Geben Sie der Projektgruppe einen aussagefähigen Namen ungleich \code{breaktroughPP}.
\item Melden Sie Ihre Projektgruppe an, unter Angabe des Namens und der Teilnehmer 
(E-Mail an \code{brosenne@informatik.uni-goettingen.de}).
\item Wählen Sie für die Prüfung einen Termin im Testzeitraum und lassen Sie sich diesen Termin bestätigen (E-Mail an \code{brosenne@informatik.uni-goettingen.de}).
\end{enumerate} \newpage
\section*{Fusionforge}
\begin{enumerate}
\item Melden Sie sich unter \code{https://fusionforge.informatik.uni-goettingen.de/} als Benutzer an.
\item Bestimmen Sie einen aus Ihrer Gruppe zum Fusionforge-Administrator, der dann ein neues Fusionforge-Projekt, unter dem Namen der Projektgruppe, für Ihre Gruppe anlegt.

Folgende Einstellungen sind vorzunehmen.
\begin{enumerate}[label=\alph*)]
\item Als Source Code Management (SCM) wählen Sie Subversion (SVN).
\item Alle Gruppenmitglieder registrieren sich im Projekt (oder werden vom Administrator registriert).
\item Das SVN-Verzeichnis ist nicht öffentlich lesbar, also nur für Mitarbeiter des Projektes zugänglich.
\item Registrieren Sie auch Ihren Tutor als Projektmitarbeiter.
\end{enumerate}
\end{enumerate} \newpage
\section*{Prüfung}
Nach der Abgabe wird das Projekt als Ganzes bewertet.

Während der Prüfung stellt jeder Teilnehmer den Teil des Projektes vor, für dessen Implementation er verantwortlich ist. Besonders interessant sind die aufgetretenen Probleme und deren Lösungen.

Neben der korrekten Umsetzung der Projektanforderungen wird gut lesbar und strukturierter Quellcode erwartet. Es sollten die Grundlagen des objektorientierten Programmentwurfs (z.B. Kapselung, Vererbung, Polymorphismus) berücksichtigt und die Möglichkeiten von Java (z.B. \emph{Java Collections Framework} ausgenutzt werden.

Jedem Teilnehmer werden Fragen zum Projekt, sowie zu Java, JavaDoc, Subversion und Ant, im Rahmen von Vorlesung und Übung, gestellt. Daraus ergibt sich für jeden Teilnehmer eine Tendenz (z.B. etwas schwächer als das Projekt insgesamt) und letztlich eine individuelle Note.
\section*{Spiel}
Realisieren Sie \textbf{BreaktroughPP} als Computerspiel in Java. Kommentieren Sie den Quellcode ausführlich. Verwenden Sie JavaDoc für das \emph{Application Programming Interface (API)} und kommentieren Sie sonst wie üblich. Verwenden Sie Ant zum automatisierten Übersetzen des Quelltextes und Erzeugen der Dokumentation.

\begin{enumerate}
\item Alle Klassen und Schnittstellen gehören zu einem Package, das mit \code{breakthroughPP} beginnt.

Bis auf die Schnittstelle \code{Viewer} dürfen die vorgegebenen Klassen und Schnittstellen des Package \code{breakthroughPP.preset}, mit Ausnahme von Kommentaren, nicht verändert werden.

Die vorgegebenen Quelltexte finden Sie in der Stud.IP-Veranstaltung unter \\
Dateien\textrightarrow Projekt\textrightarrow breakthroughPP.tgz
\item Erstellen Sie eine Spielbrett-Klasse mit folgenden Merkmalen.
\begin{itemize}
\item Ein Spielfeld mit $m \times n$ Feldern wird verwaltet. Dabei gelten $6 \le m \le 26$ und $2 \le n \le 26$.
\item Es wird eine Methode exportiert mit der Spielzüge entgegengenommen werden.
\item Gültige Spielzüge und Spielzüge die zum Ende des Spiels führen werden erkannt.

Ein leerer Zug (\code{null}) ist gültig und wird als Aufgabe des Spielers gewertet.
\item Der erste entgegengenommene Spielzug gehört immer zum roten Spieler.
\item Ein Spielzug ist ein Objekt der Klasse \code{breakthroughPP.preset.Move}, das mit Referenzen auf Objekte der Klasse \code{breaktroughPP.preset.Position} arbeitet.
\lstinputlisting{files/preset/Move.java}
\lstinputlisting{files/preset/Position.java}

\item Es wird eine Methode exportiert, die ein Objekt der Klasse\\ \code{breakthroughPP.preset.Status} zurückliefert, über die der Spielstand erfragt werden kann. Konstanten für den Status und die Spielerfarben sind in der Schnittstelle \code{breakthroughPP.preset.Setting} hinterlegt.
\lstinputlisting{files/preset/Status.java}

\item Es wird eine Methode exportiert, die -- ausgehend von der aktuellen Spielsituation -- alle möglichen gültigen Spielzüge eines Spielers zurückliefert.
\item Die Schnittstelle \code{breakthroughPP.preset.Viewable} wird implementiert.
%\lstinputlisting{files/shortened/Viewable.java}
\end{itemize}
\item Erstellen Sie eine Klasse, die die Schnittstelle \code{breakthroughPP.preset.Viewer} implementiert.
%\lstinputlisting{files/shortened/Viewer.java}
Diese Klasse soll es ermöglichen alle für das Anzeigen eines Spielbrett-Objekts nötigen Informationen zu erfragen, ohne Zugriff auf die Attribute des Spielbrett-Objekts zuzulassen.

Die Methode \code{viewer} des Spielbretts liefert ein passendes Objekt dieser Klasse.
\item Erstellen Sie eine Text-Ausgabe-Klasse mit der der aktuelle Spielstand eines Spielbretts auf die Standardausgabe geschrieben werden kann. Dazu wird ein Objekt einer Klasse, die die Schnittstelle \code{breakthroughPP.preset.Viewer} implementiert, benutzt.
\item Erstellen Sie eine Text-Eingabe-Klasse, die die Schnittstelle \\ \code{breakthroughPP.preset.Requestable} implementiert.
%\lstinputlisting{files/shortened/Requestable.java}
Die Methode \code{deliver} fordert einen Zug, in einer Zeile, von der Standardeingabe an und liefert ein dazu passendes \code{breakthroughPP.preset.Move}-Objekt zurück.
\item Alle Spieler implementieren die folgende Schnittstelle \code{breakthroughPP.preset.Player}.
\lstinputlisting{files/preset/Player.java}
\begin{itemize}[leftmargin=4em]
\item[\code{request}] \hfill \\Fordert vom Spieler einen Zug an.
\item[\code{confirm}] \hfill \\Übergibt dem Spieler im Parameter \code{boardStatus} Informationen über den letzten mit \code{request} vom Spieler gelieferten Zug.

\underline{Beispiele}
\begin{itemize}
\item Gilt \code{boardStatus.isOk()} war der letzte Zug gültig
\item Gilt \code{boardStatus.isRedWin()} war der letzte Zug gültig und der rote Spieler hat das Spiel gewonnen.
\end{itemize}
\item[\code{update}] \hfill \\Liefert dem Spieler im Parameter \code{opponentMove} den letzten Zug des Gegners und im Parameter \code{boardStatus} Informationen über diesen Zug.
\item[\code{init}] \hfill \\Initialisiert den Spieler, sodass mit diesem Spieler-Objekt ein neues Spiel mit einem Spielbrett \code{dimX} $\times$ \code{dimY} und der durch den Parameter \code{color} bestimmten Farbe, begonnen werden kann.
\end{itemize}

\underline{Weitere Anforderungen}
\begin{itemize}
\item Ein Spieler hat keine Referenz auf das Spielbrett-Objekt des Programmteils, der die Züge anfordert. Trotzdem muss ein Spieler den Spielverlauf dokumentieren, damit er gültige Züge identifizieren kann. Dazu erzeugt jeder Spieler ein eigenes Spielbrett-Objekt und setzt seine und die Züge des Gegenspielers auf diesem Brett.

Daraus können sich Widersprüche zwischen dem Status des eigenen Spielbretts und dem gelieferten Status ergeben. Das ist ein Fehler auf den mit einer Exception reagiert wird.

%NEW
Für ein problemloses Netzwerkspiel ist es nötig, dass alle Exceptions die in innerhalb der Spielerklassen geworfen werden keine Instanzen eigener Exceptionklassen sind. An jeder anderen Stelle im Spiel können eigene Exceptions frei erzeugt und geworfen werden.
\item Die Methoden der Player-Schnittstelle müssen in der richtige Reihenfolge aufgerufen werden. Eine Abweichung davon ist ein Fehler auf den mit einer Exception reagiert wird.
\end{itemize}
\item Erstellen Sie eine Interaktive-Spieler-Klasse, die die Schnittstelle \\ \code{breakthroughPP.preset.Player} implementiert.

Ein Interaktiver-Spieler benutzt ein Objekt der Text-Eingabe-Klasse um Züge vom Benutzer anzufordern.
\item Erstellen Sie eine ausführbare Klasse mit folgender Funktionalität.
\begin{itemize}
\item Es gibt eine Möglichkeit bei Ausführen der Klasse auf der Kommandozeile einen Debug-Schalter zu setzen und die Größe des Spielbretts zu übergeben.
\item Ein Spielbrett in Ausgangsposition wird initialisiert.
\item Zwei Objekte der Interaktive-Spieler-Klasse werden erzeugt und über Spieler-Schnittstellen-Referenzen angesprochen. 

Beide Spieler benutzen dasselbe Objekt der Text-Eingabe-Klasse um Züge vom Benutzer anzufordern.
\item Von den Spieler-Referenzen werden abwechselnd Züge erfragt. Gültige Züge werden bestätigt und dem jeweils anderen Spieler mitgeteilt.
\item Die gültigen Züge werden auf dem Spielbrett ausgeführt.

Ist der Debug-Schalter gesetzt, erfolgt nach jedem ausgeführten Zug eine Ausgabe des Spielbretts durch ein Objekt der Text-Ausgabe-Klasse.
\item Wenn ein Zug zum Spielende führt werden die Spieler darüber informiert und das Programm beendet.

Ist der Debug-Schalter gesetzt, wird das Ergebnis auf der Standardausgabe ausgegeben.
\end{itemize}
\item Erstellen Sie eine Computerspieler-Klasse, die die Spieler-Schnittstelle implementiert und gültige, aber nicht notwendigerweise zielgerichtete, Züge generiert. Dazu wird aus allen aktuell möglichen gültigen Spielzügen zufällig ein Zug ausgewählt.

Erweitern Sie die ausführbare Klasse so, dass Benutzer gegen den Computer und Computer gegen Computer spielen können. Das kann z.B. über Kommandozeilenparameter gesteuert werden.

\underline{Hinweis}

Java stellt für die Erzeugung von Pseudozufallszahlen die Klasse \code{java.util.Random} zur Verfügung.
\item Erstellen Sie einen weiteren Computerspieler, der zielgerichtet, entsprechend der unten beschriebenen einfachen Strategie, versucht das Spiel zu gewinnen.

Zusätzlich gelten die beiden folgenden Regeln.
\begin{enumerate}[label=\alph*)]
\item Gibt es einen Zug mit dem das Spiel gewonnen werden kann, dann wird dieser ausgeführt.
\item Ein Zug, der zur Folge hätte, dass der Gegner im nächsten Zug gewinnen könnte, wird nur ausgeführt, wenn es keine Alternative gibt.
\end{enumerate}
\item Erstellen Sie eine grafische Ein-Ausgabe-Klasse. Diese Klasse implementiert die Schnittstelle \code{breakthroughPP.preset.Requestable} und benutzt ein Objekt einer Klasse, die die Schnittstelle \code{breakthroughPP.preset.Viewer} implementiert, für eine einfache grafische Ausgabe.

Sorgen Sie dafür, dass man das Spiel Computer gegen Computer mit grafischer Ausgabe verfolgen kann, z.B. über einen Kommandozeilenparameter für eine Geschwindigkeitsreduzierung.

\underline{Hinweis}

Investieren Sie nicht zu viel Zeit in das Design, denn es wird nicht bewertet.
\item Programmieren Sie einen Netzwerkspieler mit dem sie jede Implementation der Schnittstelle \code{breakthroughPP.preset.Player} an eine andere BreakthroughPP-Implementation exportieren können.

Ein Gerüst für einen Netzwerkspieler finden Sie nachfolgend.
\begin{lstlisting}
package breakthroughPP.demo;

import java.rmi.*;
import java.rmi.server.UnicastRemoteObject;
import java.net.*;

import breakthroughPP.preset.*;

public class NetPlayer extends UnicastRemoteObject implements Player {
	public NetPlayer(Player player) throws RemoteException {
		this.player = player;
	}
	
	public Move request() throws Exception, RemoteException {
		return player.request();
	}
	
	public void update(Move opponentMove, Status boardStatus)
		throws Exception, RemoteException {
		player.update(opponentMove, boardStatus);
	}
	
	public void confirm(Status boardStatus)
		throws Exception, RemoteException {
		player.confirm(boardStatus);
	}		
	
	public void init(int dimX, int dimY, int color)
		throws Exception, RemoteException {
		player.init(dimX, dimY, color);	
	} 
	
	// ---------------------------------------------
	private Player player;
}
\end{lstlisting}
Erstellen Sie eine neue ausführbare Klasse, die Netzwerkspieler anbieten und finden kann, sowie den Spielablauf steuert und soweit nötig Objekte der Ein-Ausgabe-Klassen verwaltet. Duplizieren Sie dafür so wenig Quellcode wie möglich.

Die folgende Methode \code{offer} ist ein Beispiel wie ein Netzwerkspieler an einen Rechner \code{host}, auf dem eine RMI-Registry läuft, unter einem beliebigen Namen \code{name} angeboten werden kann.
\begin{lstlisting}
	public void offer(Player p, String host, String name) {
		try {
			Naming.rebind("rmi://" + host + "/" + name, p);
			System.out.println("Player (" + name + ") ready");
		} catch (MalformedURLException ex) {
			ex.printStackTrace();
		} catch (RemoteException ex) {
			ex.printStackTrace();
		}
	}
\end{lstlisting}
Die folgende Methode \code{find} ist ein Beispiel wie ein Spieler auf einen Rechner \code{host}, auf dem eine RMI-Registry läuft, unter einem bekannten Namen \code{name} gefunden werden kann.
\begin{lstlisting}
	public Player find(String host, String name) {
		Player p = null;
		try {
			p = (Player) Naming.lookup("rmi://" + host + "/" + name);
			System.out.println("Player (" + name + ") found");
		} catch (Exception ex) {
			ex.printStackTrace();
		}
		return p;
	}
\end{lstlisting}
\item \textbf{Optional.} Bauen Sie das Spiel weiter aus.
\begin{itemize}
\item Laden und Speichern von Spielständen
\item Erstellen Sie einen weiteren, intelligenteren Computerspieler, z.B. durch die Vorrausberechnung weiterer Züge und/oder einer besser balancierten und/oder erweiterten Bewertung.
\item \dots
\end{itemize}
\end{enumerate}

\textbf{Anforderungen an das fertige Projekt}
\begin{enumerate}
\item Per E-Mail an \code{brosenne@informatik.uni-goettingen.de} wird eine Anleitung und ein Archiv (tar, zip, etc.) ausgeliefert.
\item Das Archiv enthält den Quelltext des BreakthroughPP-Computerspiels, der sich im Rechnerpool des Instituts für Informatik übersetzen und starten lässt.

Es gibt ein Ant-Buildfile, das eine lauffähige Version des BreakthroughPP-Computerspiels, gepackt in ein Jar-File, und die vollständige API-Dokumentation erzeugen kann.
\item Die Anleitung beschreibt wie das Archiv zu entpacken ist, der Quelltext übersetzt, die API-Dokumentation erzeugt und das BreaktroughPP-Computerspiel gestartet wird. Weiterhin wird die Bedienung des Spiels beschrieben.
\end{enumerate} \newpage
%\section*{KI-Wettbewerb}
Wenn mindestens zwei Gruppen Interesse daran haben ihre Computerspieler gegeneinander antreten zu lassen wird ein KI-Wettbewerb stattfinden.

Zur Teilnahme am KI-Wettbewerb muss ein Netzwerkspieler implementiert sein.

Abhängig von der Anzahl der Teilnehmer wird ein entsprechender Spielplan erstellt. Dieser legt dann fest, welches Team auf einem Brett welcher Größe mit welcher Farbe gegen welches andere Team antritt.

Ein Matchmaking-Programm wird alle im Netzwerk zur Verfügung gestellten Spieler-Objekte verwalten und dann anhand des Spielplans mehrere Spiele starten. Vor jedem Spiel wird bei den Spieler-Objekten die Methode \code{init} mit den entsprechenden Parametern aufgerufen. Stellen Sie daher sicher, dass Ihre Spieler-Klassen beliebig oft mit neuen Werten initialisiert werden können.

Die genaue Erzeugung des Spielplans steht noch nicht fest, aber es kann davon ausgegangen werden, dass zwei konkurrierende Teams mindestens zwei Spiele gegeneinander (je eines mit den roten und blauen Steinen) spielen werden.

Es gewinnt das Team, dessen KI die meisten Spiele gewonnen hat. Im Falle eines Unentschiedens wird am Tage des Wettbewerbs vor Ort eine Lösung ausgehandelt.

Um fair zu bleiben und stundenlanges Rechnen zu unterbinden werden die folgenden Regeln gelten.
\begin{itemize}
\item Verursacht ein Spieler eine Exception oder reagiert dieser mit inkorrektem Verhalten (z.B. bei Aufruf der Methoden \code{request}, \code{confirm}, \code{update} und \code{init}) gilt dieses Spiel für den verursachenden Spieler als Niederlage.
\item Benötigt ein Spieler zu viel Zeit für einen einzelnen Zug, wird das bei mehrfacher Wiederholung als Aufgabe dieses Spielers gewertet und der andere Spieler gewinnt. Ein genaues Zeitlimit wird rechtzeitig bekannt gegeben aber eine Rechenzeit von 5 Sekunden wird auf jeden Fall möglich sein.
\item Überschreitet ein Spieler ein Limit an Gesamtzeit, wird auch das als Aufgabe gewertet.
\end{itemize}

Sie können Ihre Implementation gegen eine fertige BreakthroughPP-Implemenation testen, um zu prüfen ob Ihre KI funktioniert und ob Zeitlimits eingehalten werden. Nähere Hinweise dazu werden bekannt gegeben.

Bei Fragen oder Problemen zum KI-Wettbewerb können Sie sich gerne per E-Mail an \\
\code{dominick.leppich@stud.uni-goettingen.de} wenden. \newpage
%\section*{BreakthroughPP}
\newcommand{\myAlph}[1]{\char\numexpr`A-1+#1\relax}
\newcommand{\myalph}[1]{\char\numexpr`a-1+#1\relax}

\subsection*{Allgemein}
Die Breakthrough-Variante des Allgemeinen Programmierpraktikums (BreakthroughPP) ist ein Spiel für zwei Spieler, wobei ein Spieler mit roten und der andere mit blauen Steinen spielt.

BreakthroughPP wird auf einem Spielfeld mit quadratischen Feldern gespielt, die in $m$ Zeilen, gekennzeichnet mit aufsteigenden Nummern beginnend bei $1$, und $n$ Spalten, gekennzeichnet mit aufsteigenden Buchstaben beginnend bei $A$, angeordnet sind. Die untere linke Ecke des Spielfelds ist mit $A1$ gekennzeichnet.

\underline{Beispiel}

BreaktroughPP-Spielfeld aus $8 \times 8$ Quadraten.
\begin{center}
\begin{tikzpicture}
	\foreach \i in {0, ..., 8} {
		\draw [thin, black] (\i,0) -- (\i,8);
	}
	\foreach \i in {0, ..., 8} {
		\draw [thin, black] (0,\i) -- (8,\i);
	}
	\foreach \l in {1, ..., 8} {
		\draw node at (\l-0.5, -0.5) {\myAlph\l};
	}
	\foreach \n in {1, ..., 8} {
		\draw node at (-0.5, \n-0.5) {\n};
	}
\end{tikzpicture}
\end{center}

\subsection*{Startaufstellung}
Für $m, n$ gelten $6 \le m \le 26$ und $2 \le n \le 26$ und es gibt eine natürliche Zahl $k$ sodass \[k = \left\lfloor \frac{m + 3}{4} \right\rfloor \] 
Die unteren $k$ Zeilen von $1$ bis $k$ sind mit roten und die oberen $k$ Zeilen von $m-k+1$ bis $m$  mit blauen Steinen aufgefüllt.

Die Zeile $1$ wird hierbei als Startlinie des roten Spielers und die Zeile $m$ als Startlinie des blauen Spielers festgelegt.

\newpage
\underline{Beispiel}

Spielfelder mit $8 \times 8$, $6 \times 10$ und $14 \times 14$ Quadraten in Startaufstellung.
\begin{center}
\begin{tikzpicture}[scale=0.75]
	% Cols
	\foreach \i in {0, ..., 8} {
		\draw [thin, black] (\i,0) -- (\i,8);
	}
	% Rows
	\foreach \i in {0, ..., 8} {
		\draw [thin, black] (0,\i) -- (8,\i);
	}
	% Red
	\foreach \r in {1, ..., 3} {
		\foreach \l in {1, ..., 8} {
			\draw [color=black, fill=red] (\l-0.5,\r-0.5) circle (0.4); 		
		}
	}
	% Blue
	\foreach \r in {6, ..., 8} {
		\foreach \l in {1, ..., 8} {
			\draw [color=black, fill=blue] (\l-0.5,\r-0.5) circle (0.4); 		
		}
	}
\end{tikzpicture} \hfill \begin{tikzpicture}[scale=0.8]
	% Cols
	\foreach \i in {0, ..., 10} {
		\draw [thin, black] (\i,0) -- (\i,6);
	}
	% Rows
	\foreach \i in {0, ..., 6} {
		\draw [thin, black] (0,\i) -- (10,\i);
	}
	% Red
	\foreach \r in {1, ..., 2} {
		\foreach \l in {1, ..., 10} {
			\draw [color=black, fill=red] (\l-0.5,\r-0.5) circle (0.4); 		
		}
	}
	% Blue
	\foreach \r in {5, ..., 6} {
		\foreach \l in {1, ..., 10} {
			\draw [color=black, fill=blue] (\l-0.5,\r-0.5) circle (0.4); 		
		}
	}
\end{tikzpicture}
\end{center}
\begin{center}
\begin{tikzpicture}[scale=0.5]
	% Cols
	\foreach \i in {0, ..., 14} {
		\draw [thin, black] (\i,0) -- (\i,14);
	}
	% Rows
	\foreach \i in {0, ..., 14} {
		\draw [thin, black] (0,\i) -- (14,\i);
	}
	% Red
	\foreach \r in {1, ..., 4} {
		\foreach \l in {1, ..., 14} {
			\draw [color=black, fill=red] (\l-0.5,\r-0.5) circle (0.4); 		
		}
	}
	% Blue
	\foreach \r in {11, ..., 14} {
		\foreach \l in {1, ..., 14} {
			\draw [color=black, fill=blue] (\l-0.5,\r-0.5) circle (0.4); 		
		}
	}
\end{tikzpicture}
\end{center}

\subsection*{Spielablauf}
Der Spieler mit den roten Steinen hat den ersten Zug. Dann wird abwechselnd gezogen, einen Zug auszulassen ist nicht möglich.
\medskip

\underline{Ziehen}

Alle Steine können sowohl vertikal als auch diagonal ein Feld in Richtung der gegnerischen Startlinie auf ein freies Feld ziehen. Springen ist nicht möglich.
\medskip

\underline{Schlagen}

Ein Stein kann einen gegnerischen Stein schlagen, indem er diagonal auf ein Feld mit einem gegnerischen Stein zieht. Der gegnerische Stein wird dann vom Brett genommen. Vertikales Schlagen ist nicht möglich.
\newpage

\underline{Beispiel}

\begin{center}
\begin{tikzpicture}[scale=1]
	% Grid
	\foreach \x/\y in {2/1,3/1,4/1,1/2,2/2,3/2,4/2,5/2,1/3,2/3,3/3,4/3,5/3,1/4,2/4,3/4,4/4,5/4,2/5,3/5,4/5} {
		\draw [thin, black] (\x-0.5,\y-0.5) -- (\x+0.5,\y-0.5); 
		\draw [thin, black] (\x+0.5,\y-0.5) -- (\x+0.5,\y+0.5); 
		\draw [thin, black] (\x+0.5,\y+0.5) -- (\x-0.5,\y+0.5); 
		\draw [thin, black] (\x-0.5,\y+0.5) -- (\x-0.5,\y-0.5); 
	}
	% Arrows
	\draw [draw=black,solid,line width=0.1em,-triangle 90,fill=black] (2,2) -- (1,3);
	\draw [draw=black,solid,line width=0.1em,-triangle 90,fill=black] (2,2) -- (2,3);
	\draw [draw=black,solid,line width=0.1em,-triangle 90,fill=black] (2,2) -- (3,3);
	
	\draw [draw=black,solid,line width=0.1em,-triangle 90,fill=black] (4,4) -- (3,3);
	\draw [draw=black,solid,line width=0.1em,-triangle 90,fill=black] (4,4) -- (4,3);
	\draw [draw=black,solid,line width=0.1em,-triangle 90,fill=black] (4,4) -- (5,3);
	
	% Red
	\foreach \x/\y in {2/2} {
		\draw [color=black, fill=red] (\x,\y) circle (0.4); 
	}
	% Blue
	\foreach \x/\y in {4/4} {
		\draw [color=black, fill=blue] (\x,\y) circle (0.4); 
	}
	
	% Label
	\draw (3, 0) node {\large{a})};
\end{tikzpicture} \hfill \begin{tikzpicture}[scale=1]
	% Grid
	\foreach \x/\y in {2/1,1/2,2/2,3/2,1/3,2/3,3/3,1/4,2/4,3/4,2/5} {
		\draw [thin, black] (\x-0.5,\y-0.5) -- (\x+0.5,\y-0.5); 
		\draw [thin, black] (\x+0.5,\y-0.5) -- (\x+0.5,\y+0.5); 
		\draw [thin, black] (\x+0.5,\y+0.5) -- (\x-0.5,\y+0.5); 
		\draw [thin, black] (\x-0.5,\y+0.5) -- (\x-0.5,\y-0.5); 
	}
	% Arrows
	\draw [draw=black,solid,line width=0.1em,-triangle 90,fill=black] (2,2) -- (1,3);;
	\draw [draw=black,solid,line width=0.1em,-triangle 90,fill=black] (2,2) -- (3,3);
	
	% Red
	\foreach \x/\y in {2/2} {
		\draw [color=black, fill=red] (\x,\y) circle (0.4); 
	}
	% Blue
	\foreach \x/\y in {2/3} {
		\draw [color=black, fill=blue] (\x,\y) circle (0.4); 
	}
	
	% Label
	\draw (2, 0) node {\large{b})};
\end{tikzpicture} \hfill \begin{tikzpicture}[scale=1]
	% Grid
	\foreach \x/\y in {2/1,3/1,4/1,5/1,1/2,2/2,3/2,4/2,5/2,6/2,1/3,2/3,3/3,4/3,
	5/3,6/3,1/4,2/4,3/4,4/4,5/4,6/4,2/5,3/5,4/5,5/5} {
		\draw [thin, black] (\x-0.5,\y-0.5) -- (\x+0.5,\y-0.5); 
		\draw [thin, black] (\x+0.5,\y-0.5) -- (\x+0.5,\y+0.5); 
		\draw [thin, black] (\x+0.5,\y+0.5) -- (\x-0.5,\y+0.5); 
		\draw [thin, black] (\x-0.5,\y+0.5) -- (\x-0.5,\y-0.5); 
	}
	% Blue
	\foreach \x/\y in {2/3,3/3,5/3} {
		\draw [color=black, fill=blue] (\x,\y) circle (0.4); 
	}
	% Arrows
	\draw [draw=black,solid,line width=0.1em,-triangle 90,fill=black] (2,2) -- (1,3);
	\draw [draw=black,solid,line width=0.1em,-triangle 90,fill=black] (2,2) -- (3,3);
	\draw [draw=black,solid,line width=0.1em,-triangle 90,fill=black] (5,2) -- (4,3);
	
	% Red
	\foreach \x/\y in {2/2,5/2,6/3} {
		\draw [color=black, fill=red] (\x,\y) circle (0.4); 
	}
	
	% Label
	\draw (3.5, 0) node {\large{c})};
\end{tikzpicture}
\end{center}
\begin{enumerate}[label=\alph*)]
\item Der rote und blaue Spieler haben je drei Möglichkeiten zum Zug.
\item Der rote Spieler kann nur diagonal ziehen.
\item Der linke rote Stein kann sowohl nach links oben ziehen als auch nach rechts oben schlagen. Der rechte rote Stein kann nur nach links oben ziehen, da ein eigener Stein nicht geschlagen werden kann.
\end{enumerate}

\subsection*{Ende des Spiels}
Das Spiel ist gewonnen, wenn ein Stein die Ziellinie des gegnerischen Spielers erreicht. Ob in diesem Zug normal gezogen oder geschlagen wurde ist hierbei irrelevant.

Da in jedem Zug ein Stein bewegt werden muss und sich aufgrund der Möglichkeit diagonal zu ziehen und zu schlagen nicht alle Steine beider Spieler gegenseitig blockieren können, kann das Spiel nicht unentschieden enden.

\subsection*{Einfache Strategie}
\dots \newpage
\section*{Quellcode}
\subsection*{Preset}
Die Klasse \code{Position}, die Positionen auf dem Spielbrett beschreibt.
\lstinputlisting{files/preset/Position.java}

Die Klasse \code{Move}, die Spielzüge beschreibt.
\lstinputlisting{files/preset/Move.java}

Exceptions die beim Erzeugen von Positionen und Zügen aus Strings geworfen werden können.
\lstinputlisting{files/preset/PositionFormatException.java}
\lstinputlisting{files/preset/MoveFormatException.java} \newpage

Enumeration für Spielerfarbe, Spielertyp und Status.
\lstinputlisting{files/preset/PlayerColor.java}
\lstinputlisting{files/preset/PlayerType.java}
\lstinputlisting{files/preset/Status.java}

Die Interfaces zum Kapseln der Ein- und Ausgabe.
\lstinputlisting{files/preset/Requestable.java}
\lstinputlisting{files/preset/Viewable.java}
\lstinputlisting{files/preset/Viewer.java}

\newpage
\subsection*{Argumentparser}
Der ArgumentParser mit seiner Exception.
\lstinputlisting{files/preset/ArgumentParser.java}
\lstinputlisting{files/preset/ArgumentParserException.java}

\newpage
Ein Beispiel zur Verwendung des Parsers.
\begin{lstlisting}
package towerwarspp.demo;

import towerwarspp.preset.ArgumentParser;
import towerwarspp.preset.ArgumentParserException;

public class ArgumentParserTest {
    public static void main(String[] args) {
        try {
            ArgumentParser ap = new ArgumentParser(args);

            System.out.println("local: " + ap.isLocal());
            System.out.println("network: " + ap.isNetwork());
            System.out.println("size: " + ap.getSize());

            switch (ap.getRed()) {
                case HUMAN:
                    System.out.println("Red chose human player");
                    break;
                case RANDOM_AI:
                    System.out.println("Red chose random player");
                    break;

                // and so on..
            }
        } catch (ArgumentParserException e) {
            // Something went wrong...
            e.printStackTrace();
        }
    }
}

\end{lstlisting}

\newpage
\subsection*{Netzwerk (RMI)}
Ein Gerüst für einen Netzwerkspieler finden Sie nachfolgend.
\begin{lstlisting}
package towerwarspp.demo;

import java.rmi.*;
import java.rmi.server.UnicastRemoteObject;
import java.net.*;

import towerwarspp.preset.*;

public class NetPlayer extends UnicastRemoteObject implements Player {
	private Player player;
	
	// ---------------------------------------------
	
	public NetPlayer(Player player) throws RemoteException {
		this.player = player;
	}
	
	public Move request() throws Exception, RemoteException {
		return player.request();
	}
	
	public void update(Move opponentMove, Status boardStatus)
		throws Exception, RemoteException {
		player.update(opponentMove, boardStatus);
	}
	
	public void confirm(Status boardStatus)
		throws Exception, RemoteException {
		player.confirm(boardStatus);
	}		
	
	public void init(int size, PlayerColor color)
		throws Exception, RemoteException {
		player.init(size, color);	
	} 
}
\end{lstlisting}

Die folgende Methode \code{offer} ist ein Beispiel wie ein Netzwerkspieler an einen Rechner \code{host}, auf dem eine RMI-Registry läuft, unter einem beliebigen Namen \code{name} angeboten werden kann.
\begin{lstlisting}
	public void offer(Player p, String host, String name) {
		try {
			Naming.rebind("rmi://" + host + "/" + name, p);
			System.out.println("Player (" + name + ") ready");
		} catch (MalformedURLException ex) {
			ex.printStackTrace();
		} catch (RemoteException ex) {
			ex.printStackTrace();
		}
	}
\end{lstlisting}

Die folgende Methode \code{find} ist ein Beispiel wie ein Spieler auf einen Rechner \code{host}, auf dem eine RMI-Registry läuft, unter einem bekannten Namen \code{name} gefunden werden kann.
\begin{lstlisting}
	public Player find(String host, String name) {
		Player p = null;
		try {
			p = (Player) Naming.lookup("rmi://" + host + "/" + name);
			System.out.println("Player (" + name + ") found");
		} catch (Exception ex) {
			ex.printStackTrace();
		}
		return p;
	}
\end{lstlisting}

\end{document}